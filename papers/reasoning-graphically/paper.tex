%%%%%%%%%%%%%%%%%%%%%%% file typeinst.tex %%%%%%%%%%%%%%%%%%%%%%%%%
%
% This is the LaTeX source for the instructions to authors using
% the LaTeX document class 'llncs.cls' for contributions to
% the Lecture Notes in Computer Sciences series.
% http://www.springer.com/lncs       Springer Heidelberg 2006/05/04
%
% It may be used as a template for your own input - copy it
% to a new file with a new name and use it as the basis
% for your article.
%
% NB: the document class 'llncs' has its own and detailed documentation, see
% ftp://ftp.springer.de/data/pubftp/pub/tex/latex/llncs/latex2e/llncsdoc.pdf
%
%%%%%%%%%%%%%%%%%%%%%%%%%%%%%%%%%%%%%%%%%%%%%%%%%%%%%%%%%%%%%%%%%%%
\documentclass[runningheads]{llncs}
%\usepackage[dvips]{color}
%\usepackage{multicol}
%\usepackage{tabls}
%\usepackage{ttbox}
\usepackage{allmtt}
%\usepackage{pstricks,pst-node,pst-tree} % PS Tricks for diagrams
\setcounter{tocdepth}{3}
\usepackage{graphicx}
\usepackage{amsmath}
\usepackage{amssymb}
\usepackage{latexsym}
\usepackage{stmaryrd}
%\usepackage{mbboard}
\usepackage{xspace}
\usepackage{diagrams}
\usepackage{proof2}
\usepackage{url}


\urldef{\mailsa}\path|l.dixon@ed.ac.uk|
\urldef{\mailsb}\path|ross.duncan@comlab.ox.ac.uk|
\newcommand{\keywords}[1]{\par\addvspace\baselineskip
\noindent\keywordname\enspace\ignorespaces#1}

% -=-=-=-=-=-=-=-=-=-=-=-=-=-=-=-=-=-=-=-=-=-=-=-=-=-=-=-=-=-=-=-=-=-=-=-=-
%   Random maybe useful things...
% -=-=-=-=-=-=-=-=-=-=-=-=-=-=-=-=-=-=-=-=-=-=-=-=-=-=-=-=-=-=-=-=-=-=-=-=-
\newcommand{\cmnt}[1]{\textcolor[rgb]{ 0.8,      0,    0  }{#1}}
%\newcommand{\intabp}[2]{\parbox[t]{#1}{\raggedright{#2}\vspace{0.2cm}}}
%\newcommand{\vsmall}[1]{\footnotesize{#1}}
\newcommand{\ors}{\oplus}
\newcommand{\tensor}{\otimes}

%\newcommand{\vinterp}[1]{\lfloor\hspace{-0.27em}|\, #1\, |\hspace{-0.27em}\rfloor}
%\newcommand{\binterp}[1]{\lceil\hspace{-0.27em}|\, #1\, |\hspace{-0.27em}\rceil}
\newcommand{\vinterp}[1]{\llbracket #1 \rrbracket_v}
\newcommand{\binterp}[1]{\llbracket #1 \rrbracket_!}
\newcommand{\minterp}[1]{\llbracket #1 \rrbracket}
\newcommand{\sinterp}[1]{| #1 |}

%%-----------------------------------------------------------------
%%%% Ross's handy macros %%%%%%%%%%%%%%%%%%%%%%%%%%%%%%%%%%%%%%%%%%
%%-----------------------------------------------------------------

\newcommand{\figureline}{\rule{\textwidth}{0.5pt}}
\newcommand{\figureend}{\rule{\textwidth}{0.5pt}}

\newcommand{\dimm}{\mathrm{dim}}

% aliases
\newcommand{\infinity}{\infty}
\newcommand{\iso}{\cong}
\newcommand{\isomorphism}{\cong}

% ``semantic'' brakets
\newcommand{\denote}[1]{% ---------
\llbracket #1 \rrbracket} 
% name / coname 
\newcommand{\name}[1]{%--------
\ulcorner #1 \urcorner}
\newcommand{\coname}[1]{%
\llcorner #1 \lrcorner}

\newcommand{\sizeof}[1]{% \sizeof{x} == |x|
  \left|#1\right|}


%-------------------------------------------------------
%  Useful macros for all things categorical
%-------------------------------------------------------

\newcommand{\dom}{\operatorname{dom}}
\newcommand{\cod}{\operatorname{cod}}
\newcommand{\Tr}{\operatorname{Tr}}

% Objects of ???
\newcommand{\OBJ}[1]{\ensuremath{\mathrm{Obj}_{#1}}}
%
% Objects of {\cal ??}
\newcommand{\OBJC}[1]{\ensuremath{\mathrm{Obj}_{{\cal #1}}}}

%
% Arrows of ???
\newcommand{\ARR}[1]{\ensuremath{\mathrm{Arr}_{#1}}}
%
% Arrows of {\cal ??}
\newcommand{\ARRC}[1]{\ensuremath{\mathrm{Arr}_{{\cal #1}}}}


% Caligraphic category names
\newcommand{\catA}{\ensuremath{{\cal A}}\xspace}
\newcommand{\catB}{\ensuremath{{\cal B}}\xspace}
\newcommand{\catC}{\ensuremath{{\cal C}}\xspace}
\newcommand{\catD}{\ensuremath{{\cal D}}\xspace}
\newcommand{\catE}{\ensuremath{{\cal E}}\xspace}
\newcommand{\catF}{\ensuremath{{\cal F}}\xspace}
\newcommand{\catG}{\ensuremath{{\cal G}}\xspace}
\newcommand{\catH}{\ensuremath{{\cal H}}\xspace}
\newcommand{\catP}{\ensuremath{{\cal P}}\xspace}
\newcommand{\catQ}{\ensuremath{{\cal Q}}\xspace}

%sub scripted identity morphisms
\newcommand{\id}[1]{\ensuremath{\mathrm{id}_{#1}}}

% Boldified names of useful categories
%
\newcommand{\vectfd}[1]{% category of finite dimensional vector spaces over some field
\ensuremath{\textbf{Vect}^{\mathrm{fd}}_{#1}}\xspace}
\newcommand{\vectfdc}{% category of finite dimensional vector spaces over complexes
\vectfd{\mathbb{C}}}
\newcommand{\catRel}{% the category of sets and relations
\ensuremath{\textbf{Rel}}\xspace}
\newcommand{\catSet}{% the category of sets and functions
\ensuremath{\textbf{Set}}\xspace}
\newcommand{\catCat}{% the category of (small) categories
\ensuremath{\textbf{Cat}}\xspace}
\newcommand{\catInvCat}{% the category of involutive categories
\ensuremath{\textbf{InvCat}}\xspace}
\newcommand{\catComCl}{% the category of compact closed  categories
\ensuremath{\textbf{ComCl}}\xspace}
\newcommand{\catSComCl}{% the category of strongly compact closed  categories
\ensuremath{\textbf{SComCl}}\xspace}
\newcommand{\catGrph}{% the category of graphs
\ensuremath{\textbf{Grph}}\xspace}
\newcommand{\qubit}{% the category of qubits
\ensuremath{\textbf{Qubit}}\xspace}
\newcommand{\fdhilb}{% the category of finite dimensional hilbert space
\ensuremath{\textbf{FDHilb}}\xspace}
\newcommand{\catInvCom}{% the category of involutive categories
\ensuremath{\textbf{InvCom}}\xspace}
\newcommand{\catCom}{% the category of compact closed  categories
\ensuremath{\textbf{Com}}\xspace}
%%%% End of Ross's lovely categorcal macros--------

%%%%  use this to include graphics in the right place.
\newcommand{\inlinegraphic}[2]{
  %% todo -- make this thing calculate the height 
  %% itself based on a global scaling factor
  \dimendef\grafheight=255\dimendef\grafvshift=254
  \grafheight=#1
  \grafvshift=-0.5\grafheight
  \advance\grafvshift by 0.5ex
  \raisebox{\grafvshift}{\includegraphics[height=\grafheight]{images/#2}\xspace}
}

%%%% More handy things for me - rwd
\newcommand{\TODO}[1]{%
\typeout{WARNING!!! there is still a TODO left}
\marginpar{\textbf{!TODO: }\emph{#1}}
}
%%--------------------------------------------

\begin{document}

\mainmatter  % start of an individual contribution

% first the title is needed
\title{Reasoning with Graphs for Quantum Computation}

% a short form should be given in case it is too long for the running head
%\titlerunning{Reasoning Graphically about Quantum Computation}

% the name(s) of the author(s) follow(s) next
%
% NB: Chinese authors should write their first names(s) in front of
% their surnames. This ensures that the names appear correctly in
% the running heads and the author index.
%
\author{Lucas Dixon\inst{1} \and Ross Duncan\inst{2}}
%~\thanks{EPSRC Grants...}%
%
%\authorrunning{Lecture Notes in Computer Science: Authors' Instructions}
% (feature abused for this document to repeat the title also on left hand pages)

% the affiliations are given next; don't give your e-mail address
% unless you accept that it will be published
\institute{\mailsa, University of Edinburgh
\and \mailsb, University of Oxford
%\mailsa\\
%\mailsb\\
%\mailsc
}

%
% NB: a more complex sample for affiliations and the mapping to the
% corresponding authors can be found in the file "llncs.dem"
% (search for the string "\mainmatter" where a contribution starts).
% "llncs.dem" accompanies the document class "llncs.cls".
%
%\toctitle{Lecture Notes in Computer Science}
%\tocauthor{Authors' Instructions}
\maketitle


\begin{abstract}
  Recent graph-based formalisms of quantum computation provide an
  abstract and symbolic way to represent and simulate computations.
  However, manual manipulation of such graphs is slow and error prone.
  We present a formalism, based on compact closed categories, that
  supports mechanised reasoning about such graphs. This gives a
  compositional account of graph rewriting that preserves the
  underlying categorical semantics. Using this representation, we
  describe a generic system with a fixed logical kernel that supports
  reasoning about models of compact closed category. A salient feature
  of the system is that it provides a formal and declarative account
  of derived results that can include `ellipses'-style notation. We
  illustrate the framework by instantiating it for a graphical
  language of quantum computation and show how this can be used to
  perform symbolic computation.

  \keywords{graph rewriting, quantum computing, categorical logic,
    interactive theorem proving, graphical calculi; concerning:
    symbolic computation, automated reasoning, formal mathematics.}
\end{abstract}


\section{Introduction}
\label{sec:introduction}

Recent work in quantum computation has emphasised the use of graphical
languages motivated by the underlying logical structure of quantum
mechanics
itself~\cite{AbrCoe:CatSemQuant:2004,Selinger:dagger:2005,Coecke2005Kindergarten-Qu,Coecke2006POVMs-and-Naima,Coecke2006Quantum-Measure}.
These techniques have a number of advantages over the conventional
matrix-based approach to quantum mechanics:

\begin{itemize}
\item The visual representation abstracts over the values in the
  matrices. This removes detail that requires a lot of work for a
  human to interpret. 

\item Many properties have a natural graphical representation. For
  example, non entanglement is can be inferred from disjoint subgraphs.

\item The graphical calculus generalises to domains other than just
  vector spaces. In particular, it provides a representation for
  compact closed categories. 

\end{itemize}

The major problem with these graphical representations is the lack of
machinery for automating their manipulation. Existing approaches to
graph transformation have a different underlying semantics
corresponding to graphs in general rather than those which describe
compact closed categories. The result is that they provide an unsound
concept of rewriting for the kind of graphs we are interested in.

The main contribution of this paper is a graph-based formalism that is
suitable for representing and evaluating quantum computations. We
start by reviewing a formalism for graphical representations of
compact closed categories. We also revisit a model of quantum
computation based on this calculus. We then extend the graphical
calculus in two significant ways driven by the need to express rules
that could not be accounted for in the initial formalism. By combining
these extensions, we provide a representation that captures an
interesting and useful set of graph patterns. Notably, it can express
the Spider Theorem which is normally written using informal ellipses
notation (see Figure~\ref{fig:spider}). We provide a semantics for our
graph-based formalism in terms of the initial representation of
compact closed categories.

Using our graph-based formalism as the representational foundation, we
develop a simple logical framework for manipulating models of compact
closed categories. This has a suitable rewriting mechanism where the
axioms of the underlying object-formalism are expressed as equations
between graphs. We then present a short case study that illustrates
the framework by instantiating it for the introduced model of quantum
computation.  This shows how the framework can be used to symbolically
perform simplifications of quantum programs as well as simulate
computations.

%History of diagrammatic representation going back to ???  

%Can formalise quantum mechanics in an abstract setting using compact
%or $\dag$-compact categories.

%We will give a description of graphs in general;  a specialisation of
%this gives the structure required for compact closed  categories.  The
%graph structure suffices to give a tight representation.

%We then introduce a concrete set of generators and equations for use
%in quantum computation.  We need rewrites to capture the non-logical
%axioms of this structure.   It turns out that we can most easily express
%these equations using an infinite family of ``spider rules'';  this
%motivates the  more general notion of graph patterns.

%next section introduces the  notion of pattern graph, which is
%essentially a graph structured 

\section{Graphs and Compact Closed  Categories  }
\label{sec:mono-categ-graphs}

\subsection{Graphs}
\label{sec:graphs}

A \emph{directed graph}\footnote{ Equivalently: a directed
  graph is a functor $G$ from $\bullet \pile{\rTo\\\rTo} \bullet$ to
  \catSet; a graph morphism is then a natural transformation $f: G
  \Rightarrow H$.  } consists of a 4-tuple $(V,E,s,t)$ where $V$ and
$E$ are sets, respectively of \emph{vertices}\footnote{We will
  use the words ``vertex'' and ``node'' interchangeably.} and
\emph{edges}, and $s$ and $t$ are maps
\begin{diagram}
  E & \pile{\rTo^{\qquad s\qquad}\\\rTo_t} & V
\end{diagram}
which we call \emph{source} and \emph{target}.  The definitions
$\text{in}(v) = t^{-1}(v)$ and $\text{out}(v) = s^{-1}(v)$ express the
\emph{incoming} and \emph{outgoing} edges at a vertex $v$.  The
\emph{degree} of a vertex $v$ is $\sizeof{\text{in}(v)} +
\sizeof{\text{out}(v)}$. To distinguish between elements of different
graphs, we will use the subscript notation $G = (V_G,E_G,s_G,t_G)$.

Given graphs $G$ and $H$, a \emph{graph morphism} $f : G\to H$
consists of functions $f_E : E_G \to E_H$ and $f_V:V_G\to V_H$ such
that:

\begin{gather}
  s_H\circ f_E = f_V \circ s_G,\label{eq:graph-hom1}\\
  t_H\circ f_E = f_V \circ t_G\label{eq:graph-hom2}.
\end{gather}

\noindent These ensure that the structure of the graph is preserved by
the morphism: an edge connected to a node gets mapped to a new edge
that must be connected, in the same way, to the mapped node.

Let $f: G \to H$ be graph morphism and let $V' \subseteq V_G$. We say
that $f$ is \emph{strict for $V'$} if $\forall e \in E_H$, if $s_H(e)
\in f_V(V')$ or $t_H(e) \in f_V(V')$ then $\exists e' \in E_G$ such
that $f_E(e') = e$. Strictness ensures that there are no additional
edges in $H$.

\begin{definition}
\label{open-embedding-def}
We call a graph morphism $f$ an \emph{open embedding} for $V'$ if:
\begin{enumerate}
\item $f_E$ is injective;
\item $f_V$ restricted to $V'$ is injective; and,
\item $f$ is strict for $V'$.
\end{enumerate}
\end{definition}

\noindent The intuition behind this definition is that the subgraph of
$G$ determined by $V'$ should be preserved exactly by $f$; it should
contain $G$ without any extra incident edges into the vertices $V'$
whereas other vertices may be identified and may contain additional
vertices.

Given two graphs, $G$ and $H$, a pair of partial maps, $r_V: V_G
\rightharpoonup V_H$ and $r_E: E_G \rightharpoonup E_H$, is called a
partial graph morphism if whenever $r_E(e)$ is defined then
$r_V(s_G(e))$ and $r_V(t_G(e))$ are defined, and the restriction of
$r_E$ to its preimage satisfies equations \eqref{eq:graph-hom1} and
\eqref{eq:graph-hom2} above.

Augmented by some additional structure, graphs form a \emph{compact closed
category}.  In section \ref{sec:graph-repr-comp} we will describe this
structure, but first we review the basic properties of compact closed
categories.

\subsection{Compact Closed Categories}
\label{sec:comp-clos-categ}

\begin{definition}
\label{compactcat-def}
A strict symmetric monoidal  category
\cite{MacLane:CatsWM:1971,AspLon:CatTypStruct:1991} is called compact 
closed \cite{KelLap:comcl:1980} when each object $A$ has a chosen dual
object $A^*$, and morphisms
\begin{gather*}
  d_A : I \to A^* \otimes A \quad\quad\quad e_A : A \otimes A^* \to A
\end{gather*}
such that
\begin{align}
  A \iso A \otimes I \rTo^{\id{A} \otimes d_A} A \otimes A^* \otimes A
  \rTo^{e_A \otimes \id{A}} I \otimes A \iso A & = \id{A} \label{eq:comcl1}\\
  A^* \iso I \otimes A^* \rTo^{ d_A \otimes \id{A^*}} A^* \otimes A
  \otimes A^* \rTo^{\id{A^*} \otimes e_A} A^* \otimes I \iso A^* & =
  \id{A^*} \label{eq:comcl2}
\end{align}
\end{definition}

Every arrow $f:A\to B$ in a compact closed category \catC
has a \emph{name} and \emph{coname}:
\[
\name{f} : I \to A^* \otimes B, \qquad \coname{f} : A \otimes  B^* \to I,
\]
which are constructed as $\name{f} = (\id{A^*}\otimes f) \circ d_A$ and
$\coname{f} = e_A \circ (f \otimes \id{B^*})$.  Hence there are natural
isomorphisms $\catC(A,B) \iso \catC(I,A^*\otimes B) \iso
\catC(A\otimes B^*,I)$ making \catC monoidally closed\footnote{In
  general compact closed categories  are models of multiplicative
  linear logic where $A \multimap B$ is defined as $A^\bot \otimes B$.}.
Further,  $f$ has a dual, $f^* : B^* \to A^*$, defined by 
\[
f^* = (\id{A^*} \otimes e_B) \circ (\id{A^*}\otimes f \otimes
\id{B^*}) \circ (d_A \otimes \id{B^*})
\]
By virtue of equations \eqref{eq:comcl1} and \eqref{eq:comcl2}, $f^{**} =
f$.  Thus $(\cdot)^*$ lifts to an involutive functor
$\catC^{\text{op}} \to \catC$,  making $\catC$ equivalent to its
opposite.

\subsection{Graph Representations for Compact Closed Categories}
\label{sec:graph-repr-comp}

Graphs with certain additional structure give a representation for
compact closed categories; we now give an overview of this
construction.  The details omitted here can be found in
\cite{Duncan:thesis:2006}.  Pictorial representations are in
Fig.~\ref{fig:comcl-graphs}. We make the convention that domain of an
arrow is at the top of the picture, and its codomain is at the bottom.

\begin{figure}[t]
  \centering
  \[
  \begin{array}{lllll}
      \id{A \otimes B^*} = \inlinegraphic{3em}{comcl-id}
      &\qquad&
      d_A = \inlinegraphic{2em}{comcl-eta}
      &\qquad& 
      e_B = \inlinegraphic{2em}{comcl-epsilon} 
      \\\\
      f = \inlinegraphic{3em}{comcl-f} 
      & \qquad &
      \name{f} = \inlinegraphic{3em}{comcl-name-f}
      &\qquad &
      f^* =  \inlinegraphic{3em}{comcl-dual-f}
  \end{array}
  \]
  \caption{Compact Closed Structure as Graphs. }
  \label{fig:comcl-graphs}
\end{figure}

A \emph{concrete graph} $\Gamma$ is 5-tuple $(G, \dom\Gamma, \cod\Gamma,
<_{\text{in}(\cdot)}, <_{\text{out}(\cdot)})$ where:
\begin{itemize}
\item $G = (V,E,s,t)$ is a graph;
\item $\dom\Gamma$ and $\cod\Gamma$ are totally ordered disjoint sets of
  degree one vertices of $G$.  The union of these sets is the
  \emph{boundary} of $\Gamma$.
\item $<_{\text{in}(\cdot)}$ is a family of maps, indexed by $V$ such
  that $<_{\text{in}(v)} : \text{in}(v) \rTo^\isomorphism
  \mathbb{N}_k$ where $k = \sizeof{\text{in}(v)}$.
\item $<_{\text{out}(\cdot)}$ is a family of maps, indexed by $V$ such
  that $<_{\text{out}(v)} : \text{out}(v) \rTo^\isomorphism
  \mathbb{N}_{k'}$ where $k' = \sizeof{\text{out}(v)}$.
\end{itemize}

Since the sets $\dom\Gamma$ and $\cod\Gamma$ consist of vertices of degree
one, we can assign a polarity to each one:  $v \mapsto +$ if the edge
incident at $v$ is an incoming edge; $v \mapsto -$ otherwise.  Hence
$\cod \Gamma$ and $\dom \Gamma$ are \emph{ordered signed sets}.  Given any
ordered signed set $S$ we write $S^*$ for the same ordered set, but
the opposite signing.   Given two such sets we can define their disjoint
union $R+S$ as the disjoint union of the underlying sets, inheriting
the signing and the order from $R$ and $S$, with the convention that
$r < s$ for all $r\in R, s\in S$.

\begin{proposition}
Concrete graphs form a compact closed category whose objects are
ordered signed sets and whose arrows $f:A\to B$ are concrete graphs with
$\cod f = B$  and $\dom f = A^*$.
\end{proposition}
For each ordered signed set $A$, the identity map for $A$ has $\dom
\id{A} = A^*$ and $\cod \id{A} = A$; its underlying graph has $E = A$
and $V = A^* + A$ with $t(a) = a$ and $s(a) = a^*$.  Given a pair of
concrete graphs $f:A\to B$ and $g:B\to C$ their composition $g\circ
f:A\to C$ is constructed by merging the two graphs, erasing the
vertices of $\cod f$ and $\dom g$ (called the \emph{boundary
  vertices}), and identifying the edges previously incident at the
deleted vertices.  (Due to the opposite polarity of the domain and
codomain the edges have compatible direction.)  The tensor product on
objects $A,B$ is simply $A+B$; given $f:A\to B$, $g: C\to D$, the
graph of $f \otimes g$ is the disjoint union of the graphs of $f$ and
$g$.  The unit for the tensor is the empty set.  The morphisms $d_A :
I \to A^* \otimes A$, $e_: A \otimes A^* \to I$ have the same
underlying graph, but $\dom d = \emptyset$, $\cod d = A^*+A$, $\dom e
= A+A^*$ and $\cod e = \emptyset$.

\begin{remark}
  Given concrete graphs $f : A\to B$ and $g:B\to C$ there exists
  exactly one graph morphism  $\tau:  f \to (g\circ f)$ such that
  $\tau$ is an open embedding for the non-boundary nodes
  of $f$.   Indeed the intuition behind an open embedding, is that any
  such map picks out a subgraph which forms a well defined arrow in
  its own right.
\end{remark}

This category captures exactly the axioms for compact closed
structure, in the sense that any freely generated compact closed
category can be represented by concrete graphs.  We will consider
a collection of basic terms\footnote{See \cite{Duncan:thesis:2006} for
  a more thorough description of the nature of the terms.} $F$
whose types are vectors of some set of basic types $T$.  Then:

\begin{definition}
  A \emph{$T,F$-labelling} $\theta$ for a concrete graph $\Gamma$ is a pair of
  maps  $\theta_T : E \to T$ and $\theta_F : (V - \cod\Gamma -
  \dom\Gamma) \to F$  such that for each vertex  $v$, if
  $\text{in}(v) = \langle a_1, \ldots, a_n\rangle$ and $\text{out}(v)
  = \langle b_1, \ldots, b_m\rangle$ then 
  \[
  \theta v : \langle \theta a_1, \ldots, \theta a_n \rangle
  \to 
  \langle \theta b_1, \ldots, \theta b_m \rangle
  \]
  Call a concrete graph $\Gamma$ \emph{$T,F$-labellable} if there exists an 
  $T,F$-labelling for it; if $\theta$ is a labelling for $\Gamma$, then
 the pair $(\Gamma,\theta)$ is an \emph{$T,F$-labelled graph}.
\end{definition}

The $T,F$-labelled graphs form a compact closed category in the same
way as the the concrete graphs, subject to the further restriction
that arrows are composable only when their labellings agree.  

\begin{theorem}
  Let \catC be a compact closed category, freely generated by some set
  of arrows $F$ and ground types $T$;  then \catC is equivalent to the
  category  of $T,F$-labelled graphs.
\end{theorem}

Given a compact closed  category  \catC generated by some basic set of
operations,  the arrows of \catC have a canonical representation as
labelled graphs.  A consequence of the theorem is then that two arrows
are equal by the equations of the compact closed  structure if and
only if their graph representations are equal.

As a final remark before moving on, note that the external structure
of a vertex in a concrete graph is essentially the same as that of a
complete graph; hence one can consistently view subgraphs as vertices,
and abstract over the their internal structure.

\section{Quantum Computations as Graphs}
\label{sec:quotients-rewriting}

While compact closed categories provide a suitable setting for
reasoning about quantum computation \cite{AbrCoe:CatSemQuant:2004},
freely generated structure will not suffice:  we need additional
equations.  These equations will be expressed as rewrites rules for
graphs.  In this section we will describe a set of generators and equations
used to reason about quantum computation, and show how some of its formal
properties lead to particular issues for the rewriting machinery.

Coecke and Duncan \cite{Coecke:2008jo} propose a formal algebraic
system for quantum computation built from the following collection of
generators:  
\begin{description}
\item[Objects:] are written $Q$ and are self dual;
\item[Arrows:] there are families of arrows $X$ and $Z$:
  \begin{align*}
  &\epsilon_Z : Q \to I, &\qquad&  \epsilon_X : Q \to I,\\
  &\delta_Z : Q \to Q \otimes Q, && \delta_X : Q \to Q \otimes Q, \\
  &\alpha_Z : Q \to Q &&  \alpha_X : Q \to Q   
  \end{align*}
  where $\alpha \in [0,2\pi)$, and in addition $H:Q\to Q$.
\end{description}
To each each arrow $f : A\to B$ we assign a formal adjoint $f^\dag : B
\to A$.  Each arrow is represented as a small graph; its adjoint is
the same graph written upside down.  We use colours to denote two families:
\begin{gather*}
  \epsilon_Z = \inlinegraphic{1.5em}{epsilon} \qquad
  \delta_Z = \inlinegraphic{1.5em}{delta} \qquad
  \epsilon_Z^\dag = \inlinegraphic{1.5em}{epsilondag} \qquad
  \delta_Z^\dag = \inlinegraphic{1.5em}{deltadag} \qquad
  \alpha_Z = \inlinegraphic{1.5em}{greenalpha} \qquad
\\  
  \epsilon_X = \inlinegraphic{1.5em}{redepsilon} \qquad
  \delta_X = \inlinegraphic{1.5em}{reddelta} \qquad
  \epsilon_X^\dag = \inlinegraphic{1.5em}{redepsilondag} \qquad
  \delta_X^\dag = \inlinegraphic{1.5em}{reddeltadag} \qquad
  \alpha_X = \inlinegraphic{1.5em}{redalpha} \qquad
\end{gather*}
The $H$ arrow is denoted by \inlinegraphic{1.5em}{H}.  (The adjoints
for $H$, $\alpha_X$ and $\alpha_Z$ will be defined equationally.)  The
free compact closed category is then given by all graphs formed by
composing and tensoring these basic graphs.  Note  that since $Q$ is
self dual, that is $Q = Q^*$, we use \emph{undirected} graphs.

To model the behaviour of quantum systems certain additional equations
must be satisfied.  These are discussed in detail in
\cite{Coecke:2008jo};  we present them in graphical form in
Figure~\ref{fig:graph-quant-eqns}.


\begin{figure}[p]
  \figureline
  \begin{description}
  \item[Comonoid Laws] 
    \[
    \begin{array}{ccccccccccccccc}
      \inlinegraphic{3em}{comonoid-assoc1}  &=\;&
      \inlinegraphic{3em}{comonoid-assoc2}
      &\qquad\qquad&
      \inlinegraphic{3em}{comonoid-unit1}  &=\;&
      \inlinegraphic{3em}{comonoid-unit2}  &=\;&
      \inlinegraphic{3em}{comonoid-unit3} 
      &\qquad\qquad&
      \inlinegraphic{3em}{comonoid-comm1}  &=\;&
      \inlinegraphic{3em}{comonoid-comm2}
    \end{array}
    \]
  \item[Isometry, Frobenius, and  Compact Structure]
    \[
    \begin{array}{cccccccccccccccc}
      \inlinegraphic{3.5em}{isometry1}  &=\;&
      \inlinegraphic{3.5em}{isometry2}  
      &\qquad\qquad &
      \inlinegraphic{3em}{frobenius1}  &=\;&
      \inlinegraphic{3em}{frobenius2}  
      &\qquad\qquad&
      \inlinegraphic{2.5em}{compact1}  &=\;&
      \inlinegraphic{1.3em}{compact2}  
    \end{array}
    \]
  \item[Abelian Unitary Group]
    \[
    \begin{array}{ccccccccccccccc}
      \inlinegraphic{2.2em}{group1}  &:=\;&
      \inlinegraphic{2.2em}{group2}  &=\;&
      \inlinegraphic{2.2em}{group3} 
      &\qquad\qquad&
      \left( \inlinegraphic{2.2em}{group4}\!\right)^\dag  &=\;&
      \inlinegraphic{2.2em}{group5} 
      &\qquad\qquad&
      \inlinegraphic{3.5em}{group6}  &=\;&
      \inlinegraphic{3.5em}{group7}  &=\;&
      \inlinegraphic{3.5em}{group8} 
    \end{array}
    \]
  \item[Bilinearity] 
    \[
    \begin{array}{cccccccccccc}
      \inlinegraphic{3.5em}{alpha-commute1}  &=\;&
      \inlinegraphic{3.5em}{alpha-commute2}  &=\;&
      \inlinegraphic{3.5em}{alpha-commute3} 
    \end{array}
    \]
  \item[Bialgebra Laws] Let  $\inlinegraphic{1em}{bialgebra1} := 
    \inlinegraphic{1.5em}{bialgebra2}$;  then:
    \[
    \begin{array}{ccccccccccccccc}
      \inlinegraphic{3.5em}{bialgebra3}  &=\;&
      \inlinegraphic{3em}{bialgebra4}
      \qquad\qquad
      \inlinegraphic{2.5em}{bialgebra5}  &=\;&
      \inlinegraphic{1.5em}{bialgebra6}
      \qquad\qquad
      \inlinegraphic{2.5em}{bialgebra7}  &=\;&
      \inlinegraphic{1.5em}{bialgebra8} 
  \end{array}
  \]
\item[Group Actions]
  \[
  \begin{array}{cccccccccccccccccccccc}
    \inlinegraphic{2.5em}{group-int1} &=\;&
    \inlinegraphic{2.5em}{group-int2} 
    &\qquad\qquad&
    \inlinegraphic{2.5em}{group-int3} &=\;&
    \inlinegraphic{2em}{group-int4} 
    &\qquad\qquad&
    \inlinegraphic{2.5em}{group-int5} &=\;&
    \inlinegraphic{1.5em}{group-int6} 
    &\qquad\qquad&
    \inlinegraphic{2.5em}{group-int7} &=\;&
    \inlinegraphic{1.5em}{group-int8} 
  \end{array}
  \]
    \item[$H$ Properties]
    \[
    \begin{array}{cccccccccccccccccccc}
      \left(\inlinegraphic{1.5em}{H}\right)^\dag &=\;&
      \inlinegraphic{1.5em}{H} 
      &\qquad\qquad&
      \inlinegraphic{2.5em}{Heq1} &=\;&
      \inlinegraphic{2.5em}{Heq2}
    \end{array}
    \]
  \item[Colour Duality]
        \[
    \begin{array}{cccccccccccccccccccc}
      \inlinegraphic{2.5em}{Heq-delta} &=\;&
      \inlinegraphic{1.5em}{reddelta}
      &\qquad\qquad&
      \inlinegraphic{2em}{Heq-epsilon} &=\;&
      \inlinegraphic{1.5em}{redepsilon}
      &\qquad\qquad&
      \inlinegraphic{2.5em}{Heq-alpha} &=\;&
      \inlinegraphic{1.5em}{redalpha}
    \end{array}
    \]

  \end{description}

  
  \caption{Graphical Equations for Quantum Systems. The $(\cdot)^\dag$
    functor gives a vertical symmetry to the category, hence for every
    equation we have a second equation obtained by flipping the
    diagram upside down.  In addition, we have a ``colour duality'':
    each equation shown here gives rise to second, which is obtained
    by exchanging the two colours. The colour duality is derivable
    from the equations for involving $H$.}
  \label{fig:graph-quant-eqns}
  \figureend
\end{figure}

Consider the equations shown in Figure \ref{fig:graph-quant-eqns} which
involve only one colour:  these allow the remarkable \emph{spider theorem},
first noted in \cite{Coecke2006Quantum-Measure}, to be proved:

\begin{theorem}
  The \emph{Spider Theorem}: let $G$ be a connected graph generated
  from $\delta_Z$, $\epsilon_Z$, $\alpha_Z$ and their adjoints; then
  $G$ is totally determined by the number of inputs, the number of
  outputs, and the sum modulo $2\pi$ of the $\alpha$s which occur in
  it.
\end{theorem}

\begin{figure}[t]
$$\begin{array}{ccc}
\inlinegraphic{6em}{spider_lhs} & = & \inlinegraphic{6em}{spider_rhs}
\end{array}\qquad
\begin{array}{ccc}
\inlinegraphic{12em}{spider_lhs_patt} & \;=\; & \inlinegraphic{9em}{spider_rhs_patt}
\end{array}$$
\label{fig:spider}\caption{ [Left] An informal equation on graphs that
  expresses the Spider Theorem.  [Right] The Spider Theorem expressed
  formally using graph patterns.  The !-boxes are named $i$ and $j$.
  The variable nodes are white and named $a$ and $b$. The non-variable
  node data (the angle) is written inside the node.}

\end{figure}

Hence any connected subgraph involving nodes of only one colour,
wherever it occurs in a graph, may be collapsed to a single vertex
labelled by a single value $\alpha$, giving a ``spider''. Informally,
this can be depicted graphically as the equation in
Figure~\ref{fig:spider}. Conversely, a spider may be
arbitrarily divided into sub-spiders, provided the total in- and
out-degree is preserved, along with and the sum of the $\alpha $s.
Further, one can derive, from the spider theorem, $n$-fold versions of
many of the other equations.

Spiders offer a very intuitive way to manipulate graphs, and are far
more compact and convenient in calculations than the graphs built up
naively from the generators.  However, formalising spiders requires
moving from finite graphs, where each vertex has bounded degree, and
which are subject to a finite number of rewrite rules, to a system
where nodes may have arbitrarily many edges, and there are infinitely
many concrete rewrite rules. The desire to retain these intuitive
reasoning methods motivates the extension from concrete graphs to
\emph{graph patterns}. These will comprise the main subject of this
paper.

\section{Graphs with Variable Nodes}

In the concrete representation, the following graphs represent
different computations:

\begin{center}
\inlinegraphic{2.5em}{compact1} $=$
\inlinegraphic{1.3em}{compact2} $\qquad\quad$
\inlinegraphic{2.5em}{compact3} $=$ \inlinegraphic{1.3em}{compact4}
$\qquad\quad$ \inlinegraphic{2.5em}{compact5} $=$
\inlinegraphic{2.5em}{comonoid-unit4}
\end{center}

%\begin{center}
%\inlinegraphic{2.5em}{compact1} \quad\quad
%\inlinegraphic{2.5em}{compact3} \quad\quad
%\inlinegraphic{2.5em}{compact5}
%\end{center}

\noindent However, composition with semi-circles ($d$ and $e$ from
Definition~\ref{compactcat-def}), on the co-domain or domain, allows
an equation involving one of the above to easily lead to a derivation
corresponding to either of the other two: proving any of the above
allows a trivial derivation of the others.

To address this, we formalise a representation that abstracts over the
boundary nodes membership in the domain or co-domain. This gives rise
to a \emph{variable-node graphs}, in which boundary nodes have been
generalised to \emph{variable nodes}. The intuition of variable nodes
is that they can replaced by concrete nodes in some graph in a process
analogous to composition.

% We now define the semantics for variable-node
% graphs and then go on and introduce a way to join them.
% relationship to composition on the underlying concrete graphs.


% \subsection{Semantics by Matching as an Open Embedding}

To formalise the semantics of variable-node graphs, we define
matching, which captures the intuitive idea of a graph with variable
nodes occurring within another graph:

\begin{definition}
  \emph{Matching} is an open embedding
  (Definition~\ref{open-embedding-def}) which is strict for the
  non-variable nodes.
\end{definition}

A graph with variable nodes, $G$, can be given a formal semantics by
being interpreted as a set of concrete graphs, denoted by
$\vinterp{G}$. The interpretation is simply the set of concrete graphs
which the variable node graph matches. The above definition of
matching also applies between two variable-node graphs, in which case
we denote that $G$ matches $H$ by $G \leq_v H$.

\begin{theorem}
\label{match-interp-thm}
$G \leq_v H \Leftrightarrow \vinterp{G} \supseteq \vinterp{H}$. The
proof from left to right is a consequence of the lemma that a
composition of open embeddings is an open embedding. From left to
right, the embedding of $G \leq_v H$ composed with $\vinterp{H}$ is
thus an open embedding of $G$ into $\vinterp{H}$. Hence $\vinterp{G}
\supseteq \vinterp{H}$. From right to left, we compose the open
embedding $\vinterp{G}$, restricted to it's subset $\vinterp{H}$, with
the inverse of $\vinterp{H}$ to get $G \leq_v H$.
\end{theorem}

\begin{theorem}
\label{reflexive-match-thm}
\emph{Matching is reflexive}: $G \leq_v G$ is a trivial instance of
the same theorem for open-embeddings, where the embedding is the
identity map.
\end{theorem}

These theorems ensure that matching is what we expect it to be and
also give us the usual properties such as associativity.

% \begin{figure}[t]
% %  \scalebox{1.0}{\includegraphics{images/node-var-instance.eps}}
% *** Add figure *** 
% \label{node-variable-instances-fig}\caption{The concrete graphs $a$,
%   $b$, and $c$ are instances of $x$, a graph with variable nodes.}
% \end{figure}

% For example, in Figure~\ref{node-variable-instances-fig}, the graphs
% $a$, $b$ and $c$ are instances of the graph $x$. A graph with variable
% nodes, and without boundary nodes, is called a \emph{variable-node
%   graph}.

%% TYPES 
% Thus, unlike nodes in conrete a graph, variable nodes do not define an
% ordering on the edges that are incident. Instead, the edges that touch
% a variable node are simply a set. The type of a variable nodes is thus
% defined as a multiset rather than a tensor. Any tensor type can be
% lifted to a multiset type in the obvious manner, which we denote by
% the operation {\tt mset\_of\_tensor}. For example, {\tt
%   mset\_of\_tensor($a \tensor b \tensor a$) = $\{(a,2), (b,1)\}$}. We
% say that a tensor-type $T$ is an instance of a multiset type $M$,
% written $T \in M$, when $M \subseteq
% \mathtt{mset\_of\_tensor}(T)$. The intuition here is that when a
% variable node is instantiated to some other node, it's type will have
% at least the currently attached edges.

% An instance of a graph with variable nodes is the graph with every
% variable node replaced by a subgraph of the right type. When a graph
% has several varibale-nodes, the replacement subgraphs may be
% intersecting.

% \subsubsection{Completness and Adequacy.}

% Variable-node graphs are complete in the sense that for every concrete
% graph there is a variable node graph that can be instantiated to it:
% $\forall g.\,\exists G.\,g \in \vinterp{G}$. The witness for this
% proof is made by simply replacing all bounary nodes with variable
% nodes. 


%% not needed for this story...
%A variable node, $a$ is be said to an instance of another variable
%node $b$ when the multiset representing the type of $a$ is a superset
%of the multiset for the type of $b$. Thus, with regard to the
%underlying semantics, $a$ being a subtype of $b$ means that the set of
%instances of $a$ are a subset of the instances of $b$.

% \subsection{Variable-node Plugging}

% While concrete graphs can be composed sequentially as well as tensored
% in an adjacent manner, variable-node graphs enjoy a single but
% somewhat wilder joining operation which we call \emph{plugging}. 

% \begin{definition}
%   A \emph{plugging}, written $\pi_{p}(G,H)$, is a pair, $p$, of strict
%   partial graph morphisms, $f_G : G \rightarrow H$ and $f_H : H
%   \rightarrow G$ such that $f_G$ and $f_H$ \emph{agree}, following
%   the definition given below, on node mappings and do not map concrete
%   nodes.
% \end{definition}

% \begin{definition}
%   Two mappings $f_G$ and $f_H$ \emph{agree} when $\forall v \in
%   f_G(G).\, f_H(v) \in f^{-1}_G(v)$ and $\forall v \in f_H(H).\,
%   f_G(v) \in f^{-1}_H(v)$. This states that if a set of nodes in one
%   graph all map to a common node, $n$, in the other graph, then $n$
%   maps back to one of the nodes that maps to it.
% \end{definition}

% We let $\pi(G,H)$ abbreviate $\exists p.\,\pi_p(G,H)$. We also
% overload the notation $\pi_p(G,H)$ to denote the graph that results
% from the plugging. This is defined to be the union of the nodes and
% edges from $G$ and $H$ where the nodes and edges mapped by the pair of
% plugging function are identified. Identification of nodes instantiates
% variable nodes: when a variable node is mapped to a concrete one,
% identification results in the concrete node.

% While this is simply the intuitive way to combine graphs with variable
% nodes, the definition takes some care. In particular, it is important
% that plugging preserves incident edges of concrete nodes.
% Figure~\ref{plugging-examples-fig} presents some examples to help the
% reader get the intuition as well as understand some of the more
% subtlte cases.

% \begin{theorem}
%   \emph{Plugging is symmetric}: $\forall p.\, \exists q.\, \pi_p(G,H)
%   = \pi_q(H,G)$. Proof, given $p = (f_G,f_H)$ then $q = (f_H,f_G)$. As
%   a corollary, we get $\pi(G,H) \Leftrightarrow \pi(H,G)$.
% \end{theorem}

% \begin{theorem}
%   \emph{Plugging preserves matching}: $V \leq_v G \Rightarrow V \leq_v
%   \pi_f(G,H)$. As corrolaries we get $V \leq_v H \Rightarrow V \leq_v
%   \pi_f(G,H)$ from symmetry of plugging, $\vinterp{G} \cap \vinterp{H}
%   \subseteq \bigcup_f \vinterp{\pi_f(G,H)}$ from the definition of
%   interpretation, $G \leq_v \pi_f(G,H)$ from reflexivity of matching
%   and consequently $H \leq_v \pi_f(G,H)$ by symmetry of plugging.
% \end{theorem}

\section{!-Boxes}

A crucial feature of rules like the Spider Theorem is that an arbitrary
number of repetitions of a subgraph need to be matched. To support
this we introduce an operation, !-boxing (pronounced bang-boxing), on
graph representations. Given a graph representation, this introduces a
new notation by outlining a set of nodes. These nodes are said to be
in a !-box. Intuitively, the resulting !-boxed graph can be thought of
as representing a set of graphs with an arbitrary number of copies of
the !-boxed nodes, where every copy connects, in the same way, to the
nodes outside the !-box. The !-boxes can also be instantiated with
zero copies. This erases all edges to and from the !-box. 

% The Spider
%Theorem in Figure~\ref{fig:spider} for instance is
%represented with !-boxes in Figure~\ref{fig:spider-thm-patt}.

%Figure~\ref{fig:spider-bang-lhs-example} shows an how the left hand side of
%the spider theorem can be represented with bang boxes and illustrates
%some instances of it.

%\begin{figure}[t]
%$$\begin{array}{ccc}
%\inlinegraphic{6em}{spider_lhs_patt} & = & \inlinegraphic{6em}{spider_rhs_ex1} & , & \inlinegraphic{6em}{spider_rhs_ex2} & , & \inlinegraphic{6em}{spider_rhs_ex3} & \dots
%\end{array}$$
%\label{fig:spider-bang-lhs-example}\caption{An informal equation on graphs
%  that expresses the Spider Theorem. }
%\end{figure}

More formally, a !-box graph is a graph paired with a set !-boxes. Each
of these !-boxes is the set of nodes inside it and each !-box must be
disjoint from the others.\footnote{The disjointness condition avoids a
  slightly more complex, although also more expressive, semantics for
  overlapping node sets in the !-boxes. While such expressivity is
  interesting from a representational point of view, it is not needed
  to express the currently known rules for quantum computation.}  We
now define the concept of !-box matching which we then use to provide
a formal semantics for the !-boxed graphs.

%\textbf{Matching.} 
To formalise the intuitive notion that a !-box
represents arbitrary number of copies of the subgraph made from it's
nodes, we introduce \emph{!-box matching}. This is a binary relation
on graphs, written infix as $G \leq_! H$ when $G$ matches $H$. The set
matching !-box graphs is defined as the set closed by the following
operations on the !-boxes in a graph:

\begin{description}
  \item[copy]: copies a !-box, $b$ in a graph to produce a new graph
  with two copies of the !-boxed subgraph, $b$ is the old one and $b'$
  is the new one. The set of !-boxes in the copied graph now also
  contains the new !-box $b'$. Any edges between a node, $n$, inside
  the !-box $b$, and a node, $m$, outside it, get copied so that there
  is a new edge from $m$ to the new copy of $n$ in $b'$.

\item[drop]: simply removes the !-box, but leaves its contents in the
  graph.

\item[kill]: removes from the graph all nodes in the !-box as well as
  any incident edges.

\item[merge]: combines two !-boxes, $B_1$ and $B_2$ into a single
  larger !-box $B_1 \cup B_2$.
\end{description}

%\textbf{Semantics.} 
We give a formal semantics to a !-box graphs by
defining the !-box graph in terms of a set of graphs in the underlying
representation. We denote the interpretation of a !-box graph $G$ by
$\binterp{G}$ and say that members of $\binterp{G}$ are instances of
$G$.

\begin{definition}
  \emph{Interpretation of !-box Graphs}: $\binterp{G}$ is the subset
  of graphs matched by the !-box graph that have no !-boxes.
\end{definition}

Observe that an instance of a !-box graph is defined by pairing each
!-box with the natural number that defines how many copies are made of
it. Thus the $\binterp{G}$ is isomorphic to the set of $k$-tuples of
natural numbers, where $k$ is the number of !-boxes.

\begin{theorem}
\label{thm:bang-box-respect}
\emph{!-Matching respects !-box semantics}: $G \leq_! H
\Leftrightarrow \binterp{G} \supseteq \binterp{H}$. The proof is a
simple consequence from the definition of $\binterp{G}$ being a subset
of the graphs that match $G$.
\end{theorem}

For the purposes of this paper, the underlying graph representation of
!-boxed graphs is variable-node graphs. This gives rise to graph
patterns which we now discuss in more detail.
% The set of variable-node
% graphs for a graph $G$ with !-boxes, is denoted by $\binterp{G}$. 

\section{Graph Patterns}
\label{sec:patterns}

The representation of Compact Closed Categories as Graphs, discussed
in~\S\ref{sec:graph-repr-comp}, is too restrictive for reasoning about
quantum computation. In particular, graphical rules such the Spider
Theorem, from Figure~\ref{fig:spider}-left, are frequently needed, but
not expressible. In this section we combine the {\em !-boxes} and {\em
  variable-node} extensions to graphs. We call this representation
\emph{graph patterns}. This forms a representation that allows us to
express, in a finite way, the infinite family of concrete graphs
equations. In particular, the Spider Theorem can now be represented as
shown in Figure~\ref{fig:spider}-right. We also extend the notion of
matching for graph pattens.
%as well as present a corresponding notion of plugging. 
This provides the foundations for the rewriting machinery
in~\S\ref{sec:rewriting} which can then be used to reasoning about
quantum computation.

% \subsubsection{Completeness.}

% Graph patterns are complete in the sense that for every concrete graph
% there is a pattern that matches it. This is a trivial consequence of
% the completeness of variable-node extension to graphs, which is
% unaffected by allowing !-boxes.

% \subsubsection{Matching.}

% The purpose of graph patterns is to formally express, in a declarative
% manner, the intuitive idea of graph matching which commonly used in
% quantum computation. To do this, we combine the two extensions to the
% concrete graph representation, namely variable nodes and !-boxes, to
% provides a matching relation between graph patterns and concrete
% graphs as well as directly between graph patterns. The former relation
% gives a semantics for our system and ensures soundness, while the
% latter supports deirved equations and thus provides the theoretical
% basis for our formulation of rewriting.

%\textbf{Semantics as Concrete Matches.} 
The semantics for a graph
pattern $G$ is a set of concrete graphs denoted by $\minterp{G}$ and
define it as:

$$\minterp{G} = \{\vinterp{G'}\;.\; G' \in \binterp{G}\}$$

\noindent this simply considers every interpretation of the !-boxes to
give variable-node graphs for which we then appeal to their own
semantics. 

%\textbf{Graph Pattern Matches.} 
The specification for one pattern,
$G$, to match another one, $H$ is that it is more general with respect
to the interpretation: $\minterp{G} \supseteq \minterp{H}$. However,
graph patterns can correspond to a countably infinite number of
concrete graphs. Thus matching between graph pattern cannot be
implemented by simply unfolding all interpretations as concrete graphs
and checking the membership relation.

Fortunately, it is quite easy to provide decidable matching: the size
of the unfolding that needs to be considered can be bounded. The key
observation is that a graph $G_1$ will never match a graph with less
non-variable nodes. Thus unfolding of $G_1$ can be bounded by the
number of non-variable nodes in $G_2$. While this gives a generate and
test style algorithm, it is not efficient. The intuition for an
efficient algorithm is to search through one graph incrementally
increasing the matched part and avoids symmetries in matching.  
%The development of an efficient algorithm is ongoing work.

%non-variable nodes than another graph $G_2$
%
%can algorithm as
%
%*** add matching algo outline ***

%We can use the definition of open embedding from variable-node graphs
%as an intermediate mechanism and show that a suitable definition of
%graph pattern matching is simply the !-Box matching followed by
%variable-node matching:
%
%\begin{definition}
%  $G \leq H \Leftrightarrow \forall H' \in \binterp{H}.\,\exists G' \in \binterp{G}.\, G' \leq_v H'$
%\end{definition}
%Because variable-node matching preserves the semantics, it is trivial to
%prove that this definition meets the specification.

% \subsubsection{Graph Pattern Equality.}

% Equality of graph patterns can simply be defined as mutual matching: 

% \begin{definition}
%   $G = H \Leftrightarrow G \leq H \land H \leq G$
% \end{definition}

% \subsection{Plugging}

% We then extend the plugging
% operation on variable-node graphs to graphs with !-boxes.

% By defining graph patterns as a conservative extension of concrete
% graphs we get the relative soundness for them.

% We note that using concrete graphs as the underlying representation
% results in composition of !-box graphs to disrespect the !-box
% semantics: Theorem \ref{thm:bang-box-respects-plugging} fails to hold
% for composision.

\section{Reasoning with Graph Patterns}
\label{sec:rewriting}

In this section we describe how the graph pattern formalism can
provide a \emph{meta-level} framework for reasoning about models of
compact closed categories. Following the terminology of logical
frameworks such as Isabelle~\cite{isabelle}, we call the specification
provided by the underlying model an \emph{object-level} graph
formalism. An object-level formalisation defines a set of rules which
are treated as the axioms for the system. It also defines the data at
the nodes and edges as well as corresponding data-matching behaviour.
For it's part, the meta-level provides generic machinery to manipulate
graphs and derive new rules. We now describe the meta-level framework,
noting the conditions for a rule to be valid, and prove the systems
adequacy for rewriting.

\subsection{Equational Rules}

Axioms defined by an object-level model, as well as derived rules in
our framework, are pairs of pattern graph. The pair represents the
left and right hand sides of an equation. Rules are declarative in
that they denote a set of concrete equational rules. 

The intuitive idea of substitution with a rule is to replace a
subgraph that matches the left hand side with the right hand side.
However, not all pairs of rules are make sense with respect to the
underlying semantics. For an equation to be well defined with respect
to the compact closed structure it must not be possible to changes the
type (the boundary nodes in the domain and co-domain) of a concrete
graph by rewriting.  Mapping this restriction back to pairs of graph
patterns results in the following conditions on rules:

\begin{itemize}

\item There has to be a isomorphism between variable nodes in the left
  and right hand subgraphs. Given a matching on the left hand side,
  the target subgraph is replaced with the right hand side while
  keeping the same instantiations for the isomorphic variable nodes.

\item Rules must also define a partial injective mapping between
  !-boxes on the left and right hand sides. The intuition for this
  mapping is that the unfolding used when matching a !-box on the
  left, is applied to the mapped !-box on the right before
  replacement.

\item The interplay between !-boxes and variable nodes means that when
  a variable node appears within a !-box on one side of a rule, it must
  also appear under a mapped !-box on the other side.

\end{itemize}

Notationally, and implementationally, we annotate !-boxes and variable
nodes in a graphs with unique names. For example,
Figure~\ref{fig:spider-thm-patt} shows the Spider Theorem. In this
figure, the mapping between !-boxes is represented by !-boxes having
the same name on the left and right of a rule.  Similarly, the
isomorphism between variable nodes is captured by the set of variable
node names on the left and right hand side being equal.

\subsection{Lifting Axioms and Adequacy} 

The axioms of an object formalism come from the semantics of the
underlying system. For instance, the equations given in
Figure~\ref{fig:graph-quant-eqns} can be proved by matrix
calculations oi the underlying model. When such rules are expressed
as graph patterns, we replace the concrete representation's boundary
nodes with variable nodes. This operation is called \emph{lifting}. An
equation on graph patterns corresponds to a infinite (when their are
variable nodes) set of equations between concrete graphs. Thus we
might worry that the lifted equations express too much: they may allow
rewrites which are not true.

We call the property that the lifted representation is a conservative
extension of the initial theory \emph{adequacy}. For models of compact
closed categories, the proof of adequacy is quite simple: given an
equation between concrete graphs, $G = H$, we observe that every
instance of the lifted equation corresponds to the original equations
composed with some graph. The graph is given by the unmapped subgraph
using the open embedding from the lifted equations onto the considered
instance. Thus if $G = H$ is true, then so is every instance of it's
lifting, and thus lifting produces an adequate representation.

\subsection{Meta-Level Logic and Derived Rules}

Having defined what make a valid rule, we now present the meta-logic
of the framework. This is quite simple as it only involves dealing
with object-level equations:

\begin{center}
%% not needed, can be derived by refl + subst
%\prooftree
%A = B \in \Gamma
%\justifies
%\Gamma \vdash A = B
%\using\mbox{trivial}
%\endprooftree
%\quad\quad
\prooftree
%\mathit{valid}(A)
\justifies
\Gamma \vdash A = A
\using\mbox{refl}
\endprooftree 
\quad\quad
\prooftree
\Gamma \vdash A = B
\justifies
\Gamma \vdash B = A
\using\mbox{sym}
\endprooftree
\quad\quad
\prooftree
\Gamma \vdash A = B \quad\quad
\Gamma \vdash C = D
\justifies
\Gamma \vdash (C = D[A/B])
\using\mbox{subst}
\endprooftree
\end{center}

\noindent $\Gamma$ is the set of object-level axioms. 

We assume
that axioms in $\Gamma$ meet the validity conditions described
earlier. These rules all preserve the validity conditions on equations
and thus the system as a whole ensures only valid rules are derived.

The reflexivity rule (\emph{refl}) assumes that $A$ is a well-formed
pattern graph. This rule allows a new graph can be introduced. By then
applying the \emph{subst} rule, intermediate results are derived which
can themselves be used to rewrite other rules and conjectures.  In
this way, the system allows derived rules to provide an abbreviation
for a combination of steps.

A sets of rules can be applied automatically to simplify a graph or
simulate computation in an the object domain. For this to terminate, a
suitable left-to-right ordering on rules can be observed such as the
size of the subgraph must decrease. In \S\ref{sec:case-study} we
illustrate simulating a quantum computation.

% \subsection{Soundness and Completeness} 

% The soundness of a system that uses our meta-level rests on two
% arguments: 

% \begin{itemize}
% \item Representation of the underlying model as a compact closed
%   category with a defined set of equations.

% \item The soundness and adequacy of the axioms when expressed within
%   the compact closed category. Adequacy ensures that the encoding as a
%   graph pattern represents only the intended concrete rules. The
%   soundness of the rules is obviously needed and requires proving the
%   rule in terms of the underlying semantics.
% \end{itemize}

% The completeness of the resulting system, with respect to the intended
% underlying semantics, depends on the completeness of the set of given
% rules. Completeness is a challenging and still open problem for graph
% based models of quantum computation. Even without completeness, we
% observe in \S\ref{sec:case-study} that interesting and useful quantum
% computations can be expressed.

%\subsection{Implementation}
%\label{ssec:implementation}

% \subsection{Simplification}

% An object-level graph system is sound if their is some pair of graphs
% which can not be derived as an equation.

% To ensure soundness, it is sufficient to show that the object-level
% model is a compact-closed category and that the basic equational rules
% of the object-level can be proved in the underlying model.

% The axioms should be proved of the particular model of concrete graphs
% being considered.

% \subsubsection{Completeness.} A formalism for graphs can be shown to
% be complete with respect to the underlying semantics if the equational
% rules have been shown to be complete. 

% \subsection{Implications}


\section{A Case Study in Quantum Computation}
\label{sec:case-study}

The model of quantum computation introduced in
\S\ref{sec:quotients-rewriting} provides an object formalism for our
meta-level framework. The equations from
Figure~\ref{fig:graph-quant-eqns} are lifted to graph patterns and the
spider rule is encoded as shown in Figure~\ref{fig:spider-thm-patt}.

%Although a complete graph-based account of the underlying quantum
%model is ongoing work, these rules provide an effective and useful
%initial set.

Our model of quantum computation requires no data for the edges. The
nodes, on the other hand, are either $H$, with no additional data (a
Hadamard gate), or have an \emph{angle} and are either the $X$ or $Z$
transforms. For their part, angles are expressed as rational numbers
which correspond to the coefficient of $\pi$ in the underlying matrix.

To allow composition of rules to compute the resulting angles we give
the $X$ and $Z$ nodes an \emph{angle expression}. When a node is
within a !-box, the expression is a single \emph{angle-variable} which
gets instantiated to a new angle-variable in each of the unfoldings of
the !-box.  When a node is not within a !-box, the angle-expression is
a mapping from a set of angle-variables to the corresponding rational
coefficient. When an angle-expression contains a angle-variable within
a !-box, this is interpreted as a sum of the variables that result
from its unfolding.  This rather simple expression language has a
normal form by ordering the angle-variable by name.  Matching then
results in angle-variables being instantiated and the expressions in
all affected nodes are then re-normalised. An additional
implementation detail must also be observed for the substitution rule:
it must ensure that angle-variables in the rule being applied are
distinct from those in the expression being rewritten.


%The model of qantum computation in which we are interested is finite
%dimensional hilbert spaces. 

%\subsection{Node Expressions}
%\label{ssec:cs:node-expressions}

The quantum Fourier transform is among the most important quantum
algorithms, forming an essential part of Shor's
algorithm~\cite{Shor:PolyTimeFact:1997}, famous for providing a
polynomial time algorithm for factoring. In our graph patten calculus
this circuit becomes the top-left graph in
Figure~\ref{fig:quantum-transform}. This figure shows how computation
can be symbolically performed by rewriting with the lifted equations
from Figure~\ref{fig:graph-quant-eqns} and the graph pattern version
of the Spider Theorem.

\begin{figure}[t]
\begin{tabular}{cccc}
  & \inlinegraphic{2cm}{qft2} & = & \inlinegraphic{2cm}{qft3} \\ 
 & & & \\
 & & & \\
= & \inlinegraphic{2cm}{qft6} & = & \inlinegraphic{2cm}{qft10} \\
\end{tabular}
\label{fig:quantum-transform}\caption{An example computation of the
  Quantum Fourier Transform with inputs 1 and 0, performed
  symbolically by rewriting. }
\end{figure}



\section{Related Work}
\label{sec:relatedwork}

There are several foundational approaches to graph transformation,
including algebraic approaches~\cite{corradini97algebraic}, node-label
controlled~\cite{Graphgrammars83}, matrix
based~\cite{DBLP:conf/gg/VelascoL06}, and programmed graph
replacement~\cite{progress97}. These provide general ways of
understanding graph transformations which can then be implemented to
provide machinery for a specific application.  However, systems based
on these theories do not provide machinery for the semantics of
compact closed categories. Thus their notion of matching and
replacement do not guarantee well-typed results. Furthermore, they do
not provide machinery for rewriting of graphs with ellipses notation
which is needed to represent the Spider Theorem.

The distinctive feature of our form of graph rewriting is that the
graphs capture the structural properties of compact closed categories
and rewriting is compositional: it preserves the type of the rewritten
subgraph. However, our system can also be seen as an instantiation of
a general graph rewriting system: matching provides the embedding
information and the object-level node matching defines the application
conditions and the attribute transfer function.

% Graph Grammars...
% Other graph rewriting... 
% Matrix Based Graph Tansformation...
% Other Graph Transformation...
% Interaction nets and interaction combinators (Lafont and others)

\section{Conclusions and Further Work}
\label{sec:conclusions}

We have introduced a representation for graphs which can formally
characterise the ellipses notation used informally to represent
certain infinite families of graph rewrites, such as the Spider
Theorem.  This representation provides the foundation for a simple
meta-logic for reasoning about models of compact closed categories. We
illustrated this by providing an account of quantum computation and
showing how computation can be performed. Having developed the basic
representational machinery and shown matching to be decidable, we are
left with several exciting avenues for further research.  The most
immediate direction we are pursuing is to provide a full
implementation - only a partial one is currently
available~\footnote{\url{http://dream.inf.ed.ac.uk/projects/quantomatic}}.
Other areas of further work include considering confluence results for
sets of rewrite rules, increasing the expressiveness of the
representation for graph-patterns, and finding a complete set of
rewrite rules for the considered model of quantum computation.

% simplification ordering
% \item confluence arguments for graphs
% \item formalise algorithms in a thm prover
% \item richer graph structures - quantification over edges
% \item richer expression language in vertices
% \item completeness of rewrite rules with respect to wpHilb
% \end{itemize}

\bibliographystyle{plain}
\bibliography{all,bibfile}

\end{document}



