%%%%%%%%%%%%%%%%%%%%%%% file typeinst.tex %%%%%%%%%%%%%%%%%%%%%%%%%
%
% This is the LaTeX source for the instructions to authors using
% the LaTeX document class 'llncs.cls' for contributions to
% the Lecture Notes in Computer Sciences series.
% http://www.springer.com/lncs       Springer Heidelberg 2006/05/04
%
% It may be used as a template for your own input - copy it
% to a new file with a new name and use it as the basis
% for your article.
%
% NB: the document class 'llncs' has its own and detailed documentation, see
% ftp://ftp.springer.de/data/pubftp/pub/tex/latex/llncs/latex2e/llncsdoc.pdf
%
%%%%%%%%%%%%%%%%%%%%%%%%%%%%%%%%%%%%%%%%%%%%%%%%%%%%%%%%%%%%%%%%%%%
\documentclass[runningheads]{llncs}
%\usepackage[dvips]{color}
%\usepackage{multicol}
%\usepackage{tabls}
%\usepackage{ttbox}
\usepackage{allmtt}
%\usepackage{pstricks,pst-node,pst-tree} % PS Tricks for diagrams
%\usepackage{proof2}
\setcounter{tocdepth}{3}
\usepackage{graphicx}
\usepackage{amsmath}
\usepackage{amssymb}
\usepackage{latexsym}
\usepackage{stmaryrd}
%\usepackage{mbboard}
\usepackage{xspace}
\usepackage{diagrams}

\usepackage{url}
\urldef{\mailsa}\path|l.dixon@ed.ac.uk|
\urldef{\mailsb}\path|ross.duncan@comlab.ox.ac.uk|
\newcommand{\keywords}[1]{\par\addvspace\baselineskip
\noindent\keywordname\enspace\ignorespaces#1}

% -=-=-=-=-=-=-=-=-=-=-=-=-=-=-=-=-=-=-=-=-=-=-=-=-=-=-=-=-=-=-=-=-=-=-=-=-
%   Random maybe useful things...
% -=-=-=-=-=-=-=-=-=-=-=-=-=-=-=-=-=-=-=-=-=-=-=-=-=-=-=-=-=-=-=-=-=-=-=-=-
\newcommand{\cmnt}[1]{\textcolor[rgb]{ 0.8,      0,    0  }{#1}}
%\newcommand{\intabp}[2]{\parbox[t]{#1}{\raggedright{#2}\vspace{0.2cm}}}
%\newcommand{\vsmall}[1]{\footnotesize{#1}}
\newcommand{\ors}{\oplus}
\newcommand{\tensor}{\otimes}

\newcommand{\vinterp}[1]{\lfloor\hspace{-0.27em}|\, #1\, |\hspace{-0.27em}\rfloor}
\newcommand{\binterp}[1]{\lceil\hspace{-0.27em}|\, #1\, |\hspace{-0.27em}\rceil}
\newcommand{\minterp}[1]{\llbracket #1 \rrbracket}
\newcommand{\sinterp}[1]{| #1 |}

%%-----------------------------------------------------------------
%%%% Ross's handy macros %%%%%%%%%%%%%%%%%%%%%%%%%%%%%%%%%%%%%%%%%%
%%-----------------------------------------------------------------

\newcommand{\figureline}{\rule{\textwidth}{0.5pt}}
\newcommand{\figureend}{\rule{\textwidth}{0.5pt}}

\newcommand{\dimm}{\mathrm{dim}}

% aliases
\newcommand{\infinity}{\infty}
\newcommand{\iso}{\cong}
\newcommand{\isomorphism}{\cong}

% ``semantic'' brakets
\newcommand{\denote}[1]{% ---------
\llbracket #1 \rrbracket} 
% name / coname 
\newcommand{\name}[1]{%--------
\ulcorner #1 \urcorner}
\newcommand{\coname}[1]{%
\llcorner #1 \lrcorner}

\newcommand{\sizeof}[1]{% \sizeof{x} == |x|
  \left|#1\right|}


%-------------------------------------------------------
%  Useful macros for all things categorical
%-------------------------------------------------------

\newcommand{\dom}{\operatorname{dom}}
\newcommand{\cod}{\operatorname{cod}}
\newcommand{\Tr}{\operatorname{Tr}}

% Objects of ???
\newcommand{\OBJ}[1]{\ensuremath{\mathrm{Obj}_{#1}}}
%
% Objects of {\cal ??}
\newcommand{\OBJC}[1]{\ensuremath{\mathrm{Obj}_{{\cal #1}}}}

%
% Arrows of ???
\newcommand{\ARR}[1]{\ensuremath{\mathrm{Arr}_{#1}}}
%
% Arrows of {\cal ??}
\newcommand{\ARRC}[1]{\ensuremath{\mathrm{Arr}_{{\cal #1}}}}


% Caligraphic category names
\newcommand{\catA}{\ensuremath{{\cal A}}\xspace}
\newcommand{\catB}{\ensuremath{{\cal B}}\xspace}
\newcommand{\catC}{\ensuremath{{\cal C}}\xspace}
\newcommand{\catD}{\ensuremath{{\cal D}}\xspace}
\newcommand{\catE}{\ensuremath{{\cal E}}\xspace}
\newcommand{\catF}{\ensuremath{{\cal F}}\xspace}
\newcommand{\catG}{\ensuremath{{\cal G}}\xspace}
\newcommand{\catH}{\ensuremath{{\cal H}}\xspace}
\newcommand{\catP}{\ensuremath{{\cal P}}\xspace}
\newcommand{\catQ}{\ensuremath{{\cal Q}}\xspace}

%sub scripted identity morphisms
\newcommand{\id}[1]{\ensuremath{\mathrm{id}_{#1}}}

% Boldified names of useful categories
%
\newcommand{\vectfd}[1]{% category of finite dimensional vector spaces over some field
\ensuremath{\textbf{Vect}^{\mathrm{fd}}_{#1}}\xspace}
\newcommand{\vectfdc}{% category of finite dimensional vector spaces over complexes
\vectfd{\mathbb{C}}}
\newcommand{\catRel}{% the category of sets and relations
\ensuremath{\textbf{Rel}}\xspace}
\newcommand{\catSet}{% the category of sets and functions
\ensuremath{\textbf{Set}}\xspace}
\newcommand{\catCat}{% the category of (small) categories
\ensuremath{\textbf{Cat}}\xspace}
\newcommand{\catInvCat}{% the category of involutive categories
\ensuremath{\textbf{InvCat}}\xspace}
\newcommand{\catComCl}{% the category of compact closed  categories
\ensuremath{\textbf{ComCl}}\xspace}
\newcommand{\catSComCl}{% the category of strongly compact closed  categories
\ensuremath{\textbf{SComCl}}\xspace}
\newcommand{\catGrph}{% the category of graphs
\ensuremath{\textbf{Grph}}\xspace}
\newcommand{\qubit}{% the category of qubits
\ensuremath{\textbf{Qubit}}\xspace}
\newcommand{\fdhilb}{% the category of finite dimensional hilbert space
\ensuremath{\textbf{FDHilb}}\xspace}
\newcommand{\catInvCom}{% the category of involutive categories
\ensuremath{\textbf{InvCom}}\xspace}
\newcommand{\catCom}{% the category of compact closed  categories
\ensuremath{\textbf{Com}}\xspace}
%%%% End of Ross's lovely categorcal macros--------

%%%%  use this to include graphics in the right place.
\newcommand{\inlinegraphic}[2]{
  %% todo -- make this thing calculate the height 
  %% itself based on a global scaling factor
  \dimendef\grafheight=255\dimendef\grafvshift=254
  \grafheight=#1
  \grafvshift=-0.5\grafheight
  \advance\grafvshift by 0.5ex
  \raisebox{\grafvshift}{\includegraphics[height=\grafheight]{images/#2}\xspace}
}

%%%% More handy things for me - rwd
\newcommand{\TODO}[1]{%
\typeout{WARNING!!! there is still a TODO left}
\marginpar{\textbf{!TODO: }\emph{#1}}
}
%%--------------------------------------------

\begin{document}

\mainmatter  % start of an individual contribution

% first the title is needed
\title{Reasoning Graphically about Quantum Computation}

% a short form should be given in case it is too long for the running head
%\titlerunning{Reasoning Graphically about Quantum Computation}

% the name(s) of the author(s) follow(s) next
%
% NB: Chinese authors should write their first names(s) in front of
% their surnames. This ensures that the names appear correctly in
% the running heads and the author index.
%
\author{Lucas Dixon\inst{1} \and Ross Duncan\inst{2}~\thanks{EPSRC Grants...}%
}
%
%\authorrunning{Lecture Notes in Computer Science: Authors' Instructions}
% (feature abused for this document to repeat the title also on left hand pages)

% the affiliations are given next; don't give your e-mail address
% unless you accept that it will be published
\institute{\mailsa, University of Edinburgh
\and \mailsb, University of Oxford
%\mailsa\\
%\mailsb\\
%\mailsc
}

%
% NB: a more complex sample for affiliations and the mapping to the
% corresponding authors can be found in the file "llncs.dem"
% (search for the string "\mainmatter" where a contribution starts).
% "llncs.dem" accompanies the document class "llncs.cls".
%
%\toctitle{Lecture Notes in Computer Science}
%\tocauthor{Authors' Instructions}
\maketitle


\begin{abstract}
  Systems of quantum computation are typically represented by large
  matrix compositions. However, recent graph-based formalisms offer an
  approach which exposes the structure of these systems in a clearer
  and more accessible way. Although these provide a significantly
  simpler way of reading and presenting quantum computations, manual
  manipulation of such graphs is slow and error prone. We
  present a formalism that supports a mechanised reasoning about such
  graphs. This involves a compositional account of graph rewriting
  that preserves the underlying categorical semantics. We present a
  system, which has been implemented, with a fixed logical kernel and
  support for derived rules, which allows the model of quantum
  computation  to be specified declarativley.
  Soundness of the 
  system is proved with respect to the underlying model and
  completeness is an open problem.

  \keywords{graph rewriting, quantum computing, categorical
    logic, interactive theorem proving, graphical calculi}
\end{abstract}


\section{Introduction}
\label{sec:introduction}

Want to do reasoning about quantum computation by manipulating graphs

History of diagrammatic representation going back to ???  

CAn formalise quantum mechanics in an abstract setting using compact
or $\dag$-compact categories.

We will give a description of graphs in general;  a specialisation of
this gives the structure required for compact closed  categories.  The
graph structure suffices to give a tight representation.

We then introduce a concrete set of generators and equations for use
in quantum computation.  We need rewrites to capture the non-logical
axioms of this structure.   It turns out that we can most easily express
these equations using an infinite family of ``spider rules'';  this
motivates the  more general notion of pattern graphs.

next section introduces the  notion of pattern graph, which is
essentially a graph structured 

\section{Graphs and Compact Closed  Categories  }
\label{sec:mono-categ-graphs}

\subsection{Graphs}
\label{sec:graphs}

A \emph{directed graph}\footnote{
Equivalently:  a directed graph is a functor $G$ from  $\bullet
\pile{\rTo\\\rTo} \bullet$ to \catSet;  a graph morphism is then a natural
transformation  $f: G \Rightarrow H$.
} consists of a 4-tuple $(V,E,s,t)$ where $V$
and $E$ are sets, respectively of \emph{vertices}\footnote{We will use
  the words ``vertex'' and ``node'' interchangeably.} and \emph{edges},
and $s$ and $t$ are maps 
\begin{diagram}
  E & \pile{\rTo^{\qquad s\qquad}\\\rTo_t} & V
\end{diagram}
which we call \emph{source} and \emph{target}.  Define sets
$\text{in}(v) = t^{-1}(v)$ and $\text{out}(v) = s^{-1}(v)$ of \emph{incoming}
and \emph{outgoing} edges at a vertex $v$.  
The \emph{degree} of a vertex $v$ is $\sizeof{\text{in}(v)} +
\sizeof{\text{out}(v)}$. 

Given graphs $G$ and $H$, a \emph{graph morphism} $f : G\to H$ consists of
functions $f_E : E_G \to E_H$ and $f_V:V_G\to V_H$ such that:
\begin{gather*}
  s_H\circ f_E = f_V \circ s_G,\\
  t_H\circ f_E = f_V \circ t_G.
\end{gather*}
\TODO{different kinds of morphism?}

Augmented by some additional structure, graphs form a compact closed
category.  In section \ref{sec:graph-repr-comp} we will describe this
structure, but first we review the concept of compact closed  category.

\subsection{Compact Closed Categories}
\label{sec:comp-clos-categ}



\subsection{Graph Representations for Compact Closed Categories}
\label{sec:graph-repr-comp}

Graphs with certain additional structure give a
representation for compact closed  categories; we now give an overview
of this construction.  The details omitted here can be found in
\cite{Duncan:thesis:2006}. 

A \emph{concrete graph} $\Gamma$ is 5-tuple $(G, \dom\Gamma, \cod\Gamma,
<_{\text{in}(\cdot)}, <_{\text{out}(\cdot)})$ where:
\begin{itemize}
\item $G = (V,E,s,t)$ is a graph;
\item $\dom\Gamma$ and $\cod\Gamma$ are totally ordered disjoint sets of
  degree one vertices of $G$.
\item $<_{\text{in}(\cdot)}$ is a family of maps, indexed by $V$ such
  that $<_{\text{in}(v)} : \text{in}(v) \rTo^\isomorphism
  \mathbb{N}_k$ where $k = \sizeof{\text{in}(v)}$.
\item $<_{\text{out}(\cdot)}$ is a family of maps, indexed by $V$ such
  that $<_{\text{out}(v)} : \text{out}(v) \rTo^\isomorphism
  \mathbb{N}_{k'}$ where $k' = \sizeof{\text{out}(v)}$.
\end{itemize}

Since the sets $\dom\Gamma$ and $\cod\Gamma$ consist of vertices of degree
one, we can assign a polarity to each one:  $v \mapsto +$ if the edge
incident at $v$ is an incoming edge; $v \mapsto -$ otherwise.  Hence
$\cod \Gamma$ and $\dom \Gamma$ are \emph{ordered signed sets}.  Given any
ordered signed set $S$ we write $S^*$ for the same ordered set, but
the opposite signing.   Given two such sets we can define their disjoint
union $R+S$ as the disjoint union of the underlying sets, inheriting
the signing and the order from $R$ and $S$, with the convention that
$r < s$ for all $r\in R, s\in S$.

\begin{proposition}
Concrete graphs form a compact closed category whose objects are
ordered signed sets and whose arrows $f:A\to B$ are concrete graphs with
$\cod f = B$  and $\dom f = A^*$.
\end{proposition}
For each ordered signed set $A$, the identity map for $A$ has
$\dom \id{A} = A^*$ and $\cod \id{A} = A$; its underlying
graph has $E = A$ and $V = A^* + A$ with $t(a) = a$ and $s(a) =
a^*$.  Given a pair of concrete graphs $f:A\to B$ and $g:B\to C$ their
composition $g\circ f:A\to C$ is constructed by merging the two
graphs, erasing the vertices of $\cod f$ and $\dom g$, and identifying
the edges previously incident at the deleted vertices.  (Due to the
opposite polarity of the domain and codomain the edges have compatible
direction.)  The tensor product on objects $A,B$ is simply $A+B$;
given $f:A\to B$, $g: C\to D$, the graph of $f \otimes g$ is the
disjoint union of the graphs of $f$ and $g$.  The unit for the tensor
is the empty set.  The morphisms $d : I \to A \otimes A^*$,
$e: A^* \otimes A \to I$ have the same underlying graph, but $\dom d =
\emptyset$, $\cod d = A+A^*$, $\dom e = A^*+A$ and $\cod e = \emptyset$.

This category captures exactly the axioms for compact closed
structure, in the sense that any freely generated compact closed
category can be represented by concrete graphs.  We will consider
a collection of basic terms\footnote{See \cite{Duncan:thesis:2006} for
  a more thorough description of the nature of the terms here} $F$
whose types are vectors of some set of basic types $T$.  Then:

\begin{definition}
  A \emph{$T,F$-labelling} $\theta$ for a concrete graph $\Gamma$ is a pair of
  maps  $\theta_T : E \to T$ and $\theta_F : (V - \cod\Gamma -
  \dom\Gamma) \to F$  such that for each vertex  $v$, if
  $\text{in}(v) = \langle a_1, \ldots, a_n\rangle$ and $\text{out}(v)
  = \langle b_1, \ldots, b_m\rangle$ then 
  \[
  \theta v : \langle \theta a_1, \ldots, \theta a_n \rangle
  \to 
  \langle \theta b_1, \ldots, \theta b_m \rangle
  \]
  Call a concrete graph $\Gamma$ \emph{$T,F$-labellable} if there exists an 
  $T,F$-labelling for it; if $\theta$ is a labelling for $\Gamma$, then
 the pair $(\Gamma,\theta)$ is an \emph{$T,F$-labelled graph}.
\end{definition}

The $T,F$-labelled graphs form a compact closed category in the same
way as the the concrete graphs, subject to the further restriction
that arrows are composable only when their labellings agree.  

\begin{theorem}[\cite{Duncan:thesis:2006}]
  Let \catC be a compact closed category  freely generated by some set
  of arrows $F$ and ground types $T$;  then \catC is equivalent to the
  category  of $T,F$-labelled graphs.
\end{theorem}

Given a compact closed  category  \catC generated by some basic set of
operations,  the arrows of \catC have a canonical representation as
labelled graphs.  A consequence of the theorem is then that two arrows
are equal by the equations of the compact closed  structure if and
only if their graph representations are equal.

As a final remark before moving on, note that the external structre of vertex
in a concrete graph is esentially the same as that of a complete
graph;  hence one can consistently view subgraphs as vertices, and
abstract over the their internal structure.

\section{Quotients and Rewriting}
\label{sec:quotients-rewriting}

Few examples of interesting compact closed categories are freely
generated.  In practice we work with categories presented as
generators and \emph{equations}.  For 
representing quantum computations, Coecke and Duncan
\cite{Coecke:2008jo} propose the following collection of generators:
\begin{description}
\item[Objects] the self dual object $Q$;
\item[Arrows] Two families of arrows:
  \begin{align}
  &\epsilon_Z : Q \to I, &\qquad&  \epsilon_X : Q \to I,\\
  &\delta_Z : Q \to Q \otimes Q, && \delta_X : Q \to Q \otimes Q, \\
  &\alpha_Z : Q \to Q &&  \alpha_X : Q \to Q   
  \end{align}
  where $\alpha \in [0,2\pi)$, and in addition $H:Q\to Q$.
\end{description}
We represent these two families, and their adjoints, by colours in our
graphical notation:
\begin{gather*}
  \epsilon_Z = \inlinegraphic{1.5em}{epsilon} \qquad
  \delta_Z = \inlinegraphic{1.5em}{delta} \qquad
  \epsilon_Z^\dag = \inlinegraphic{1.5em}{epsilondag} \qquad
  \delta_Z^\dag = \inlinegraphic{1.5em}{deltadag} \qquad
  \alpha_Z = \inlinegraphic{1.5em}{greenalpha} \qquad
\\  
  \epsilon_X = \inlinegraphic{1.5em}{redepsilon} \qquad
  \delta_X = \inlinegraphic{1.5em}{reddelta} \qquad
  \epsilon_X^\dag = \inlinegraphic{1.5em}{redepsilondag} \qquad
  \delta_X^\dag = \inlinegraphic{1.5em}{reddeltadag} \qquad
  \alpha_X = \inlinegraphic{1.5em}{redalpha} \qquad
\end{gather*}
The $H$ arrow is denoted by \inlinegraphic{1.5em}{H}.  The free
compact closed category  is then given by all graphs formed by
composing and tensoring these basic graphs.

To model the behaviour of quantum systems certain additional equations
must be satisfied.  Each family $(\delta, \epsilon)$ must form a
\emph{classical structure}
\cite{Coecke2006Quantum-Measure,Coecke2006POVMs-and-Naima}:  the pair
$(\delta, \epsilon)$  should form a cocommuative comonoid, the pair
$(\delta^\dag,\epsilon^\dag)$ must form a commutative monoid, and
together they must satisfy the isometry and frobenius equations. In
addition the family $\alpha$ must form an abelian group with $\alpha^\dag
= -\alpha$.  These equations are presented graphically below for the
green family;  the red vertices obey all the same equations
\begin{description}
\item[Comonoid Laws] 
\[
\begin{array}{ccccccccccccccc}
  \inlinegraphic{3em}{comonoid-assoc1}  &=\;&
  \inlinegraphic{3em}{comonoid-assoc2}
  &\qquad\qquad&
  \inlinegraphic{3em}{comonoid-unit1}  &=\;&
  \inlinegraphic{3em}{comonoid-unit2}  &=\;&
  \inlinegraphic{3em}{comonoid-unit3} 
  &\qquad\qquad&
  \inlinegraphic{3em}{comonoid-comm1}  &=\;&
  \inlinegraphic{3em}{comonoid-comm2}
\end{array}
\]
\item[Monoid Laws] 
\[
\begin{array}{ccccccccccccccc}
  \inlinegraphic{3em}{monoid-assoc1}  &=\;&
  \inlinegraphic{3em}{monoid-assoc2}
  &\qquad\qquad&
  \inlinegraphic{3em}{monoid-unit1}  &=\;&
  \inlinegraphic{3em}{monoid-unit2}  &=\;&
  \inlinegraphic{3em}{monoid-unit3} 
  &\qquad\qquad&
  \inlinegraphic{3em}{monoid-comm1}  &=\;&
  \inlinegraphic{3em}{monoid-comm2}
\end{array}
\]
\item[Isometry]
\[
\begin{array}{ccc}
  \inlinegraphic{3.5em}{isometry1}  &=\;&
  \inlinegraphic{3.5em}{isometry2}  
\end{array}
\]
\item[Frobenius]
  \[
  \begin{array}{ccccc}
    \inlinegraphic{3em}{frobenius1}  &=\;&
    \inlinegraphic{3em}{frobenius2} &=\;&
    \inlinegraphic{3em}{frobenius3}
  \end{array}
  \]
\item[Compact Structure] 
  \[
  \begin{array}{ccccccc}
    \inlinegraphic{2.5em}{compact1}  &=\;&
    \inlinegraphic{1.3em}{compact2}  
    &\qquad\qquad&
    \inlinegraphic{2.5em}{compact3}  &=\;&
    \inlinegraphic{1.3em}{compact4}  
  \end{array}
  \]
  \item[Abelian Unitary Group]
    \[
    \begin{array}{ccccccccccccccc}
      \inlinegraphic{2.2em}{group1}  &:=\;&
      \inlinegraphic{2.2em}{group2}  &=\;&
      \inlinegraphic{2.2em}{group3} 
      &\qquad\qquad&
      \left( \inlinegraphic{2.2em}{group4}\!\right)^\dag  &=\;&
      \inlinegraphic{2.2em}{group5} 
      &\qquad\qquad&
      \inlinegraphic{3.5em}{group6}  &=\;&
      \inlinegraphic{3.5em}{group7}  &=\;&
      \inlinegraphic{3.5em}{group8} 
    \end{array}
    \]
  \item[Bilinearity] 
    \[
    \begin{array}{cccccccccccc}
      \inlinegraphic{3.5em}{alpha-commute1}  &=\;&
      \inlinegraphic{3.5em}{alpha-commute2}  &=\;&
      \inlinegraphic{3.5em}{alpha-commute3} 
      &\qquad\qquad&
      \inlinegraphic{3.5em}{alpha-commute4}  &=\;&
      \inlinegraphic{3.5em}{alpha-commute5}  &=\;&
      \inlinegraphic{3.5em}{alpha-commute6} 
    \end{array}
    \]
\end{description}
Additionally, the two families satisfy equations which make them a
\emph{scaled bialgebra}.
\begin{description}
\item[Bialgebra Laws] Let  $\inlinegraphic{1em}{bialgebra1} := 
      \inlinegraphic{1.5em}{bialgebra2}$;  then:
  \[
  \begin{array}{ccccccccccccccc}
      \inlinegraphic{3.5em}{bialgebra3}  &=\;&
      \inlinegraphic{3em}{bialgebra4}
      \qquad\qquad
      \inlinegraphic{2.5em}{bialgebra5}  &=\;&
      \inlinegraphic{1.5em}{bialgebra6}
      \qquad\qquad
      \inlinegraphic{2.5em}{bialgebra7}  &=\;&
      \inlinegraphic{1.5em}{bialgebra8} \\ \\
      \inlinegraphic{3.5em}{bialgebra9}  &=\;&
      \inlinegraphic{3em}{bialgebra10}
      \qquad\qquad
      \inlinegraphic{2.5em}{bialgebra11}  &=\;&
      \inlinegraphic{1.5em}{bialgebra12}
      \qquad\qquad
      \inlinegraphic{2.5em}{bialgebra13}  &=\;&
      \inlinegraphic{1.5em}{bialgebra14} 
  \end{array}
  \]
\end{description}
Last group of equations

Statement and Discussion of spider law.

The need to formalise these equations leads naturally to a rewriting
theory 

\section{Graphs with Variable Nodes}

Variable nodes generalise the concept of a boundary nodes. Unlike
boundary-nodes in the concrete graph, they are not classified into
domain and co-domain. The intuition of a variable node is that it can
be replaced by any subgraph of the right type. Thus, unlike nodes in
conrete a graph, variable nodes do not define an ordering on the edges
that are incident. Instead, the edges that touch a variable node are
simply a set. The type of a variable nodes is thus defined as a
multiset rather than a tensor. Any tensor type can be lifted to a
multiset type in the obvious manner, which we denote by the operation
{\tt mset\_of\_tensor}. For example, {\tt mset\_of\_tensor($a \tensor
  b \tensor a$) = $\{(a,2), (b,1)\}$}. We say that a tensor-type $T$
is an instance of a multiset type $M$, written $T \in M$, when $M
\subseteq \mathtt{mset\_of\_tensor}(T)$. The intuition here is that
when a variable node is instantiated to some other node, it's type
will have at least the currently attached edges.

\begin{figure}[t]
%  \scalebox{1.0}{\includegraphics{images/node-var-instance.eps}}
*** Add figure *** 
\label{node-variable-instances-fig}\caption{The concrete graphs $a$,
  $b$, and $c$ are instances of $x$, a graph with variable nodes.}
\end{figure}

A graph with variable nodes, $G$, can be given a formal semantics by
being interpretated as a set of concrete graphs which is denoted by
$\vinterp{G}$. We refer to members of this set as instances. An
instance of a graph with variable nodes is the graph with every
variable node replaced by a subgraph of the right type. When a graph
has several varibale-nodes, the replacement subgraphs may be
intersecting. For example, in
Figure~\ref{node-variable-instances-fig}, the graphs $a$, $b$ and $c$
are instances of the graph $x$. A graph with variable nodes, and
without boundary nodes, is called a \emph{variable-node graph}. Such
graphs are complete in the sense that for every concrete graph there
is a variable node graph that can be instantiated to it: $\forall
g.\,\exists G.\,g \in \vinterp{G}$. The witness for this proof is made
by simply replacing all bounary nodes with variable nodes.

%% not needed for this story...
%A variable node, $a$ is be said to an instance of another variable
%node $b$ when the multiset representing the type of $a$ is a superset
%of the multiset for the type of $b$. Thus, with regard to the
%underlying semantics, $a$ being a subtype of $b$ means that the set of
%instances of $a$ are a subset of the instances of $b$.

Below, we give a more formal and constuctive account the semnatics by
introducing a notion of an \emph{open embedding} and note that this
corresponds to the intuitive idea of graph-matching. We then go on and
introduce a way to join graphs with variable-nodes and note it's
relationship to composition on the underlying concrete graphs. 

\subsection{Matching and Open Embedding}

\begin{definition}
  An \emph{open embedding} from $G$ to $H$ a strict embedding from the
  non-variable subgraph of $G$ into the non-variable subgraph of $H$
  and a mapping from the variable nodes in $G$ ...
\end{definition}


\subsection{Variable-node Plugging}

While concrete graphs can be composed sequentially as well as tensored
in an adjacent manner, variable-node graphs enjoy a single but
somewhat wilder joining operation which we call \emph{plugging}. 

\begin{definition}
  A \emph{plugging}, written $\pi_{p}(G,H)$, is a pair, $p$, of
  strict partial homomorpihc mappings, $f_G : G \Longrightarrow H$ and
  $f_H : H \Longrightarrow G$ such that $f_G$ and $f_H$ do not map
  concrete nodes and for the nodes that are mapped \emph{agree}
  following the definition given below.
\end{definition}

\begin{definition}
  Two mappings $f_G$ and $f_H$ \emph{agree} when $\forall v \in
  f_G(G).\, f_H(v) \in f^{-1}_G(v)$ and $\forall v \in f_H(H).\,
  f_G(v) \in f^{-1}_H(v)$. Intuitively this captures the idea a node
  is one function is always mapped to one the nodes that maps to it in
  the other function.
\end{definition}

We let $\pi(G,H)$ abbreviate $\exists p.\,\pi_p(G,H)$. We also
overload the notation $\pi_p(G,H)$ to let it denote the graph that
results from the plugging. This is defined to be the union of the
nodes and edges from $G$ and $H$ where the nodes and edges mapped by
the pair of plugging function are identified. Identification of nodes
instantiates variable nodes: when a variable node is mapped to a
concrete one, the result is a concrete node.

While this is simply the intuitive way to combine graphs with variable
nodes, the definition takes some care. In particular, it is important
that plugging preserves the types of concrete nodes and results in
subtypes in variable nodes. Figure~\ref{plugging-examples-fig}
presents some examples to help the reader get the intuition as well as
understand some of the more subtlte cases.

\begin{theorem}
  \emph{Plugging is symmetric}: $\forall p.\, \exists q.\, \pi_p(G,H)
  = \pi_q(H,G)$. Proof follows simply: given $p = (f_G,f_H)$, $q =
  (f_H,f_G)$. As a corrolary, we get $\pi(G,H) \Leftrightarrow
  \pi(H,G)$.
\end{theorem}

\begin{theorem}
  \emph{Match after plugging}: $G \leq \pi_f(G,H)$. And as a corrolary
  we get $H \leq \pi_f(G,H)$ by symmetry of plugging. 
\end{theorem}

\begin{theorem}
  \emph{Plugging preserves matching}: $V \leq G \Rightarrow V \leq
  \pi_f(G,H)$. As corrolaries we get $V \leq H \Rightarrow V \leq
  \pi_f(G,H)$ from symmetry of plugging and $\vinterp{G} \cap
  \vinterp{H} \subseteq \bigcup_f \vinterp{\pi_f(G,H)}$ from the
  definition of interpretation. 
\end{theorem}




\section{!-Boxes}

A crucial feature of rules like the Spider Law is that an arbirary
number of repetitions of a subgraph to be matched. To support this we
introduce an operation, !-boxing, on graph representations. Given a
graph representation, this introduces a new notation by outlining
(called !-boxing) a set of chosen nodes. Intuitively, the resulting
!-boxed graph can be thought of as representing a set of graphs with
an arbirary number of copies of the !-boxed nodes, where every copy
has the connects to the same nodes outside the !-box. The !-boxes can
also be instantiated with zero copies which is erases all edges from
the !-box. See Figure~\ref{bang-box-ex1-fig} for a simple example.

More formaly, a graph with !-boxes is simply a graph paired with a set
!-boxes. Each !-box is just the set of nodes inside it and each !-box
must be disjoint from the others.\footnote{The disjointness condition
  avoids a slightly more complex, although also more expressive,
  semantics for overlaping node sets in the !-boxes. While such
  expressivity is interesting from a representational point of view,
  it is not needed to express the currently known rules of quantum
  computation.}

For the purposes of this paper, the underlying graph representation of
!-boxed graphs is variable-node graphs. The set of variable-node
graphs for a graph $G$ with !-boxes, is denoted by $\binterp{G}$. We
now define the concept of !-box matching which we use to provide a
semantics for the !-boxed graphs. 

\subsection{!-Matching.}

The set of !-box graphs matched by a !-box graph is defined as the set
closed by the following operations:

\begin{definition}
  \emph{copy} copies a !-box in a graph to produce a new graph with
  two copies of the boxed subgraph, with a second box around the newly
  copied nodes. Any edges between a node inside the !-box, $n$ and one
  outside it $m$ get copied so that there is a new edge from $m$ to
  the new copy of $n$.
\end{definition}

\begin{definition}
  \emph{drop} simply removes the box, but leaves its contents in the
  graph.
\end{definition}

\begin{definition}
  \emph{kill} removes from the graph all nodes in the !-box set as
  well as any incident edges to those nodes.
\end{definition}

\begin{definition}
  \emph{merge} combines two !-boxes, $B_1$ and $B_2$ into a single
  larger !-box $B_1 \cup B_2$.
\end{definition}

Like the notation for variable-node graphs, we use $G \leq H$ to
denote that $G$ matches $H$. 

\subsection{!-Box Semantics.} 

We give a formal semantics to !-boxing a representation of graphs by
defining the !-box graph in terms of a set of graphs in the
underlying representation. We denote the interpretation of a graph $G$
by $\binterp{G}$ and say that mebemrs of $\binterp{G}$ are instances
of $G$. 

\begin{definition}
  \emph{Interpretation of !-box Graphs}: $\binterp{G}$ is the subset
  of graphs matched by the !-box graph that have no !-boxes.
\end{definition}

Observe that an instance of a !-box graph is defined by pairing each
!-box with the natural number that defines how many copies are made of
it. Thus the $\binterp{G}$ is isomorphic to the set of $k$-tuples of
natural numbers, where $k$ is the number of !-boxes.

\begin{theorem}
  \emph{!-Matching respects !-box semantics}: $G \leq H
  \Leftrightarrow \binterp{G} \supseteq \binterp{H}$. The proof is a
  simple consequence from the defintion of $\binterp{G}$ being a
  subset of the graphs that match $G$.
\end{theorem}


\section{Graph Patterns}
\label{sec:patterns}

The representation of Compact Closed Categories as Graphs, discussed
in~\S\ref{graph-repr-comp}, is too restrictive for reasoning about
quantum computation. In particular, pictures such the Spider Law, in
Figure~\ref{spider-law-fig} are frequently needed, but not
expressible. Such pictures are used to denote an infinite family of
concrete graphs that can be rewritten. In this section we combine the
{\em !-boxes} with {\em variable nodes} graph extensions to form a
representation that allows us to finitely express these infinite
families of concrete graphs. We call this extension \ref{graph
  patterns}. The Spider Law can now be represented by the equation in
Figure~\ref{graph-pattern-spider-law-fig}. We also extend the notion
of matching for graph pattens as well as present a corresponding
notion of plugging. This provides the foundations for the rewriting
machinery in~\S\ref{rewriting} which can then be used to reasoning
about quantum computation.

\subsection{Matching}

The purpose of graph patterns is to formally express, in a declarative
manner, the intuitive idea of graph matching commonly used in for
Quantum computation. To do this, we combine the two extensions to the
concrete graph representation, namely variable nodes and !-boxes. This
provides a matching relation between graph patterns and concrete
graphs as well as directly between graph patterns. The former relation
gives a semantics for our system and ensures soundness, while the
latter supports deirved equations and thus provides the theoretical
basis for our formulation of rewriting.

\subsubsection{Concrete Matches.}

The semantics for a graph pattern $G$ in terms of a set of concrete
graphs, is denoted by $\minterp{G}$. We say that members of this set
are \emph{matched} by $G$. Formally, the set of matches are defined as
follows:

$\minterp{G} = (\vinterp{G'} | G' \in \binterp{G})$

Graph patterns are complete in the sense that for every concrete graph
there is a pattern that matches it. This is a trivial consequence of
the completeness of variable-node extension to graphs, which is
unaffected by allowing !-boxes.

\subsubsection{Graph Pattern Matches.}

The definition of one pattern matching another one is simply that it
is more general. Formally, a graph pattern, $G$, matches another graph
pattern, $H$, when $\minterp{G} \supseteq \minterp{H}$. This provides
the specification for a matching algorithm which we implement of
in~\S\ref{sec:rewriting}. To make such a proof easier, we extend
embedding and plugging of variable-node graphs to graph patterns. 

Our representation has been carefully designed so that this concept
corresponds to what we call a \emph{open embedding} of $G$ into
$H$. 


We then extend the plugging
operation on variable-node graphs to graphs with !-boxes.


\subsection{Plugging}



*** 

This is the basis for our implementation and is thus also the basis
for its correctness argument. 




\section{Graph Pattern Rewriting}
\label{sec:rewriting}

\subsection{Rules}
\subsection{Unification}
\subsection{Substitution}


\section{Node Expressions}
\label{sec:node-expressions}


\section{Related Work}
\label{sec:relatedwork}

Initially one might be tempted to think that the usual notion of
subgraph might surfice for expressing matching, however it allows
additional edges on nodes where, because of the tensorial semantics of
our graphs, we do not. Our work provides a restriction of the usual
subgraph definition that is sound for Compact Closed Categories.

Link Graphs and their extention to BiGraphs also encapsualte a very
different kind of semantics where they allow edges to go to multiple
nodes. An interesting area of further work would be to consider our
graph patterns for these, and other, graph based formalisms.

Graph Grammars...

Other graph rewriting... 

Matrix Based Graph Tansformation...

Other Graph Transformation...

Interaction nets and interaction combinators (Lafont and others)

\section{Conclusions}
\label{sec:conclusions}

Ideas for further work
\begin{itemize}
\item simplification ordering
\item confluence arguments for graphs
\item formalise algorithms in a thm prover
\item richer graph structures - quantification over edges
\item richer expression language in vertices
\item completeness of rewrite rules with respect to wpHilb
\end{itemize}

\bibliographystyle{plain}
\bibliography{all}
\bibliography{bibfile}

\end{document}



