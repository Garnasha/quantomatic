\documentclass{article}
\usepackage{amsmath}
\usepackage{amssymb}
\usepackage{color}
\usepackage{enumitem}
\usepackage{fullpage}
\usepackage{graphicx}

\DeclareMathOperator{\dom}{dom}
\DeclareMathOperator{\im}{im}

\DeclareMathOperator{\N}{N}
\DeclareMathOperator{\PN}{PN}
\DeclareMathOperator{\W}{W}

\newcommand{\TODOitem}{%
\typeout{WARNING!!! there is still a TODO left}
\color{blue}\item
}
\newcommand{\TODO}[1]{%
\typeout{WARNING!!! there is still a TODO left}
\marginpar{\textbf{!TODO: }\emph{#1}}
}
\newcommand{\TODOinline}[1]{%
\typeout{WARNING!!! there is still a TODO left}
{\color{blue}\textbf{!TODO: }#1}
}

\begin{document}

\section{Normalisation}
\label{sec:normalisation}

We normalise both the pattern and target graphs by ensuring that
\begin{itemize}
\item each wire connected to a node-vertex at both ends has exactly two wire-vertices on it
\item boundary vertices are adjacent to either node-vertices or another boundary vertex
\item circles (wires in a loop with no node-vertices) have exactly two wire-vertices on them
\end{itemize}

\section{Bare Wires and Circles}
\label{sec:wires-and-circles}

Two forms of graph component require special attention: \emph{circles}, which are wires forming a closed loop with no node-vertices, and \emph{bare wires}, which are graph components equivalent to isolated points.  A circle is normalised to two wire-vertices with two edges between them, one in each direction.  A bare wire is normalised to an input connected directly to an output.  Normalisation will ensure that any wire-vertex not in a circle or bare wire will be adjacent to exactly one node-vertex.

Circles can only ever match against other circles, and all matchings for a given rule and target that are identical except on circles will produce isomorphic graphs after rewriting.  Therefore, all that is relevant when matching circles is the number of circles in the pattern and the target.  Without $!$-boxes, it is sufficient to check that there are at least as many circles in the target as in the pattern, and pick an arbitrary matching of circles; this does not require any branching.  Circle matching is therefore always done first in any round of conrete matching.  More circle matching may be necessary after expanding $!$-boxes in the pattern, as this may produce more circles.

Bare wires are more complicated, as they can match any wire in the target, not just bare wires.  This can create a huge amount of branching.  We therefore leave wire matching until right at the end of the matching process, so that the ``hard'' part of the matching is not repeated unnecessarily.  When matching a bare wire, we need to choose an unmatched edge in the target graph (one where either the edge's source or target are not matched, or where the edge's source and target are matched by an output and an input respectively) and split it into three edges joined with two wire-vertices.  The bare wire is then matched against these two new wire-vertices and the edge between them.


\section{Match State}
\label{sec:match-state}

For a string graph $G$, let $N(G)$ be the set of node-vertices and $W(G)$ the wire-vertices.

Let $L$ be the pattern and $G$ the target. A match state consists of:

\begin{itemize}
    \item $U_C \subseteq W(L)\times W(L)$ : a set of unmatched circles, as pairs of wire-vertices
    \item $U_W \subseteq W(L)$ : a set of unmatched (non-circle) wire-vertices
    \item $U_N \subseteq N(L)$ : a set of unmatched node-vertices
    \item $P \subseteq N(L)$ : a set of partially-matched node-vertices
    \item $P_S \subseteq P$ : a set of partially-matched node-vertices scheduled for re-checking
    \item $m : V_L \rightarrow V_G$ : a partial function describing the match so far
\end{itemize}

Throughout the entire procedure, in addition to the invariants implicit in the typing of the components of the match state, we maintain the following global invariants:
\begin{enumerate}
  \renewcommand{\theenumi}{(\arabic{enumi})}
  \renewcommand{\labelenumi}{\theenumi}
  \item \label{enum:g-inv-inj} $m$ is injective and respects vertex types
  \item \label{enum:g-inv-local-iso} $P \subseteq \dom(m)$ and for any node-vertex $v \in \dom(m)\setminus P$, all vertices adjacent to $m(v)$ in $G$ are in the image of $m$
  \item \label{enum:g-inv-edge-match} For any $v,w \in \dom(m)$, if there is an edge from $v$ to $w$ in $L$, then there is an edge of the same type from $m(v)$ to $m(w)$ in $G$
\end{enumerate}

Matching is complete when $m$ is total and $P$ is empty.  Providing we maintain the above invariants and ensure that $L$ and $G$ are simple graphs, $m$ will induce a unique graph monomorphism $\hat m:L \rightarrow G$, and this will be a local isomorphism.

That $\hat m$ exists and is the unique morphism from $L$ to $G$ that restricts to $m$ can be seen by construction: invariant \ref{enum:g-inv-edge-match} ensures that a mapping consistent with $m$ can be found for each edge in $L$ and simplicity of $G$ ensures that there is only one such mapping.  Invariant \ref{enum:g-inv-inj} ensures the vertex-function (ie: $m$) is injective, and so simplicity of $L$ ensures that the edge function is also injective, and invariants \ref{enum:g-inv-inj} and \ref{enum:g-inv-edge-match} ensure that both functions respect typing, and hence $\hat m$ is a monomorphism.  Invariant \ref{enum:g-inv-local-iso} ensures that $\hat m$ will be a local isomorphism.

Providing the match state satisfies the above invariants, $P$ is empty and $m$ is total everywhere except on bare wires, the bare wire matching procedure given in section \ref{sec:wires-and-circles} maintains the invariants and makes $m$ total on $V_L$.


\section{Inner Loop: Concrete Matching}
\label{sec:inner-loop}

The goal of the concrete matching step is to match all of the concrete vertices in the graph. If a node-vertex is not connected to a !-box, its complete neighbourhood will also be matched. If it is connected to one or more !-boxes, only some if its neighbourhood is matched. We refer to these as \textit{partially matched} node-vertices. A node-vertex $v$ such that the neighbourhood of $m(v)$ is totally covered is called \textit{completely matched}.

The concrete matching routine is given a function ``doomed'', which returns true if a match of a concrete node-vertex is impossible. For !-box matching this returns true if the (concrete) neighbourhood of $v$ contains fewer in-edges (resp. out-edges) than that of $m(v)$ and there are no !-boxes connected to $v$ by an in-edge (resp. out-edge). It also takes a match state, subject to the following preconditions:

\subsection{Preconditions}
\label{sec:il-precond}

These preconditions are actually invariants of the algorithm.

\begin{enumerate}
  \renewcommand{\theenumi}{(\roman{enumi})}
  \renewcommand{\labelenumi}{\theenumi}
  \item \label{enum:il-inv-norm} $L$ and $G$ are both normalised
  \item \label{enum:il-inv-inv} The match state satisfies the global invariants
  \item \label{enum:il-inv-unmatched} $U_C\cap\dom(m) = U_W\cap\dom(m) = U_N\cap\dom(m) = \varnothing$ (ie: the vertices in the unmatched sets are not part of the existing matching)
  \item \label{enum:il-inv-reachable} Each vertex in $U_W$ is adjacent to something in $U_N\cup P_S$ (ensures we can always reach wire vertices in $U_W$ by starting from a node-vertex in $U_N$ or $P_S$)
  \item \label{enum:il-inv-Ps} If $v \in P$ and $N_G(m(v)) \subseteq \im(m)$, $v \in P_S$ (everything that is completely matched and in $P$ is also in $P_S$)
  \item \label{enum:il-inv-wire-uw} If $v \in U_W$ and $w$ is a wire-vertex adjacent to $v$ in $L$, $w \in U_W$.
  \item \label{enum:il-inv-wire-connected} If $v \in \dom(m)$ is a wire-vertex, then any vertices it is adjacent to are also in $\dom(m)$
  \item \label{enum:il-inv-circles} $U_C$ contains only circles, and $U_W$ contains no wire-vertices that are in circles
\end{enumerate}

\subsection{Algorithm}

A note on the algorithm presentation: red wires (with labels starting ``foreach'') indicate a branching point.  A branch is created for each possibility.  If there are no possibilities, the current branch dies.  So, for example, if we take a vertex from $U_N$ and there are no possible matchings for it in the target graph, we kill off the current branch.

Note also that when the algorithm calls for a matching, it is implicitly assumed that the image of the matching should not intersect with the image of $m$.

\begin{center}
  \includegraphics[height=16cm]{concrete_part.pdf}
\end{center}

\subsection{Postconditions}
These are true for the match state that results from each successful branch:
\begin{enumerate}
  \renewcommand{\theenumi}{(\Roman{enumi})}
  \renewcommand{\labelenumi}{\theenumi}
  \item \label{enum:il-postcond-inv} The match state satisfies the global invariants
  \item \label{enum:il-postcond-no-unmatched} $U_C = U_W = U_N = P_S = \varnothing$ (every vertex marked for matching was handled)
  \item \label{enum:il-postcond-all-matched} $\dom(m)$ is the union of the initial states of $U_C$, $U_W$, $U_N$ and $\dom(m)$ (we have matched exactly what we were asked to)
  \item \label{enum:il-postcond-P} $P$ is exactly the set of vertices in $\dom(m)$ whose image is adjacent to a vertex not in $\im(m)$ ($P$ contains exactly the partially-matched vertices)
\end{enumerate}

Postconditions \ref{enum:il-postcond-inv} and \ref{enum:il-postcond-all-matched} state that the algorithm has completed the requested work.  Postcondition \ref{enum:il-postcond-no-unmatched} just states that we left the match state ``clean''; it still satisfies all the preconditions and repeating the algorithm without altering the state further would do nothing.

\subsection{Correctness}

First, we note that our data structure constraints are never broken: we never add anything to $U_C$, $U_W$ or $U_N$, so these cannot contain objects of the wrong type.  When we add things to $P$, they are always either taken from $U_N$ or are the end of a wire that does not terminate in a vertex in $U_W$; normalisation and precondition \ref{enum:il-inv-wire-uw} guarantee that the vertex is a node-vertex in this case.  Finally, when removing vertices from $P$ they are removed from $P_S$ simultaneously, and when adding vertices to $P_S$ they are added to $P$ at the same time.

\subsubsection{Invariants}

The preconditions (section \ref{sec:il-precond}) are maintained throughout the algorithm as invariants:

\begin{enumerate}[label=(\roman*),ref=(\roman*)]
  \item $L$ and $G$ are both normalised

  This is trivially maintained as we alter neither graph.
  \item The match state satisfies the global invariants

  See section \ref{sec:il-global-inv}.
  \item $U_C\cap\dom(m) = U_W\cap\dom(m) = U_N\cap\dom(m) = \varnothing$

  Nothing is ever added to $U_C$, $U_W$ or $U_N$.  There are three places where vertices are added to $m$.  The first is when matching circles, when we match the contents of $U_C$ and set $U_C$ to be empty.  We cannot add anything from $U_W$ at this point, due to invariant \ref{enum:il-inv-circles}, and we cannot add anything from $U_N$, since that contains only node-vertices.  The second is when $P_S$ is empty and we take a vertex from $U_N$ and add its matching to $m$, which clearly maintains the invariant.  In the final case all the wire-vertices added to $m$ are removed from $U_W$ at the same time and if a node-vertex is added to $m$ it has already been removed from $U_N$.  In both the final two cases, $U_C$ is already empty.
  \item Each vertex in $U_W$ is adjacent to something in $U_N\cup P_S$

  We never add anything to $U_W$, so we only have to consider the removal of vertices from $U_N$ and $P_S$.  When we remove a vertex from $U_N$, we always immediately add it to $P_S$, and we only remove a vertex from $P_S$ when there are no adjacent vertices in $U_W$.
  \item If $v \in P$ and $N_G(m(v)) \subseteq \im(m)$, $v \in P_S$ (everything that is completely matched and in $P$ is also in $P_S$)

  When we add anything to $P$, we also add it to $P_S$, and when we remove anything from $P_S$, we remove it from $P$ if it is completely matched.
  \item If $v \in U_W$ and $w$ is a wire-vertex adjacent to $v$ in $L$, $w \in U_W$.

  When we remove a vertex from $U_W$, we remove the entire wire that it is on.
  \item If $v \in \dom(m)$ is a wire-vertex, then any vertices it is adjacent to are also in $\dom(m)$

  We never remove anything from $\dom(m)$ (invariant \ref{enum:il-inv-extend}).  There are only two places we add wire-vertices: when matching circles and when matching wires connected to a partially matched node-vertex.  In the former case, each wire-vertex we add to $m$ is adjacent to exactly one other vertex, which we also add to $m$.  In the latter case, we add the entire wire, including the end-point not already in $m$, to $m$.  So both cases preserve the invariant.
  \item $U_C$ contains only circles, and $U_W$ contains no wire-vertices that are in circles

  This is trivially maintained as we never add anything to either set.
\end{enumerate}

To these, we add an additional invariant:
\begin{enumerate}[resume,label=(\roman*),ref=(\roman*)]
  \item \label{enum:il-inv-extend} $m$ restricts to its initial state (it is only ever extended, never changed in any other way)

  This is clearly true at the start of the algorithm.  When matching circles, we only ever add vertices that are in $U_C$, which is disjoint from $\dom(m)$ by invariant \ref{enum:il-inv-unmatched}.  When $P_S$ is empty we only ever add a single node-vertex, which is drawn from $U_N$ (which, again by \ref{enum:il-inv-unmatched}, is disjoint from $\dom(m)$).

  When exploring wires connected to a vertex in $P$, normalisation ensures that if the wire terminates with a wire-vertex, it must be the only wire-vertex on the wire, which means that we only add a single vertex to $m$, and that vertex is drawn from $U_W$, which is disjoint from $\dom(m)$.  Otherwise, it must terminated with a node-vertex and the wire must contain exactly two wire-vertices.  The first is in $U_W$, and so the second (which it is adjacent to) must also be in $U_W$ by invariant \ref{enum:il-inv-wire-uw}, and hence neither can be in $\dom(m)$ by invariant \ref{enum:il-inv-unmatched}.  The only other vertex added to $m$ is the terminating node-vertex.  If this is in $P$, it is only added if the matchings agree (which doesn't change anything).  If not, the node-vertex cannot be in $\dom(m)$ by global invariant \ref{enum:g-inv-local-iso} and the fact that its image is adjacent to a wire-vertex not in the image of $m$ (since we only consider partial matchings whose image does not intersect with the image of $m$).
\end{enumerate}


\subsubsection{Global Invariants}
\label{sec:il-global-inv}


\begin{enumerate}
  \renewcommand{\theenumi}{(\arabic{enumi})}
  \renewcommand{\labelenumi}{\theenumi}
  \item $m$ is injective and respects vertex types

  This follows from the fact that it is only ever extended with valid partial matchings whose image is disjoint from the full subgraph defined by the image of $m$.
  \item $P \subseteq \dom(m)$ and for any node-vertex $v \in \dom(m)\setminus P$, all vertices adjacent to $m(v)$ in $G$ are in the image of $m$

  The first part is maintained as any vertex added to $m$ is added to $P$.  There are three points where we extend $m$.  When we match circles, the partial match that is added encompasses the entire graph component connected to the vertices we add to $m$.  In both the other cases, the vertex is added to $P$, and so the invariant is trivially satisfied.

  \item For any $v,w \in \dom(m)$, if there is an edge from $v$ to $w$ in $L$, then there is an edge of the same type from $m(v)$ to $m(w)$ in $G$

  There are three points where we alter $m$, and we will consider them independently.  When we match circles, we only add wire-vertices, so the invariant is trivially maintained.  When $P_S$ is empty and we extend $m$ with a matching from an element of $U_N$, the invariant is also maintained as none of the adjacent (wire-)vertices can be in $\dom(m)$, by invariant \ref{enum:il-inv-wire-connected}.

The final case we need to consider is where we add a wire connected to something in $P$.  If the matched wire ended in a wire-vertex, only one vertex was added, and the only incident edge to that vertex has a corresponding edge in the target.  Otherwise, a wire terminated with a node-vertex was added.  All the incident edges of the two wire-vertices on the wire have corresponding edges in the target (by the matching), so all we have to consider are the incident edges of the node-vertex.  If it was already in $\dom(m)$, we already know the invariant holds.  Otherwise, none of the incident wire-vertices could have already been in $\dom(m)$, by invariant \ref{enum:il-inv-wire-connected}, and so global invariant \ref{enum:g-inv-edge-match} must be maintained.
\end{enumerate}


\subsubsection{Termination}
\label{sec:termination}

For termination of the inner loop exploring the wires incident to a node-vertex $v$, we consider the variant $n = |U_W\cap N_L(v)|$, where $N_L(v)$ is the neighbourhood of $v$ in $L$.  In each of these branches, $n$ decreases by either $1$ or $2$, and nothing is ever added to $U_W$ (and $N_L(v)$ is fixed).  So this loop must terminate.

The loop working through $P_S$ has two branches: when it is not empty and when it is not.  Either may add more things to $P_S$.  We will take as variant $m = 2|U_N| + |P_S|$.  When this is $0$, the algorithm will terminate.  Otherwise, if $P_S$ is empty, $U_N$ will have one vertex removed and added to $P_S$, decreasing $m$ by one.  If $P_S$ is not empty, one vertex will be removed from $P_S$, decreasing $m$ by one.  Every vertex that is added to $P_S$ in this loop is taken from $U_N$, further decreasing $m$.  As a result, $m$ will decrease by at least one with every iteration of the loop.

\subsubsection{Postconditions}

\begin{enumerate}
  \renewcommand{\theenumi}{(\Roman{enumi})}
  \renewcommand{\labelenumi}{\theenumi}
  \item The match state satisfies the global invariants

  See section \ref{sec:il-global-inv}.
  \item $U_C = U_W = U_N = P_S = \varnothing$

  $U_C$ is emptied by the circle-matching step and is never added to.  $P_S$ and $U_N$ must be empty when we return the match due to the termination condition ($m = 0$ in section \ref{sec:termination}).  Invariant \ref{enum:il-inv-reachable} guarantees that if $U_N$ and $P_S$ are empty, then so must $U_W$ be.
  \item $\dom(m)$ is the union of the initial states of $U_C$, $U_W$, $U_N$ and $\dom(m)$

  Every time we remove a vertex from $U_C$, $U_W$ or $U_N$, we add it to $m$.  Conversely, every time we add a vertex to $m$, it must be drawn from one of these sets (given invariant \ref{enum:il-inv-wire-uw}).  This, coupled with postcondition \ref{enum:il-postcond-no-unmatched} and the fact that we only ever extend $m$, gives us postcondition \ref{enum:il-postcond-all-matched}.

  \item $P$ is exactly the set of vertices in $\dom(m)$ whose image is adjacent to a vertex not in $\im(m)$

  Global invariant \ref{enum:g-inv-local-iso} gives us most of what we need.  We just need that if $v$ is completely matched, it is not in $P$.  But this is guaranteed by invariant \ref{enum:il-inv-Ps} and the fact that $P_S = \varnothing$.
\end{enumerate}

\subsubsection{Completeness}

\TODOinline{all possible ways of satisfying the postconditions are found}
\section{Testing Wrapper: Concrete Matching Only}
\label{sec:test-wrapper}

For testing purposes, it is straight-forward to wrap the above procedure to just do concrete matching.  When we populate the match state, we do it in the following manner:
\begin{itemize}
\item $U_C$ will contain all circles
\item $U_W$ will contain all other wire vertices, except those forming bare wires (boundary vertices connected to another boundary vertex)
\item $U_N$ will contain all node vertices
\item $P$, $P_S$ and $m$ will be empty
\end{itemize}

This will ensure the preconditions for the concrete matching inner-loop will be satisfied (assuming normalisation).  The only precondition that is not trivially satisfied is \ref{enum:il-inv-reachable}, but this follows from the fact that normalisation ensures that the only wire-vertices not adjacent to a node-vertex are those in circles and bare wires.

\begin{center}
  \includegraphics[height=12cm]{concrete_wrapper.pdf}
\end{center}

The postconditions of the concrete matching inner-loop, together with how we created the initial match state, ensure that a successful termination of this algorithm produces a valid matching, since $m$ will be total after matching bare wires and the global invariants from section \ref{sec:match-state} are maintained.

\section{Full Wrapper: Lazy !-Box Expansion}
\label{sec:full-wrapper}

This algorithm performs lazy $!$-box expansion to find all possible matchings of a pattern graph against a string graph.  In order for the algorithm to be deterministic, we disallow pattern graphs with empty $!$-boxes or $!$-boxes containing only bare wires, as these may be expanded any number of times and still match any graph that contains a wire.

Since we are only matching pattern graphs on to concrete graphs, it suffices to restrict $!$-box operations to only two: KILL and EXPAND (COPY, then DROP the new box).

\begin{center}
  \includegraphics[height=16cm]{pattern_part.pdf}
\end{center}


\subsection{Termination}

Each step terminates, and there is only one loop.  As a variant, we take
\[ n = |V_G| + |V_L\setminus\dom(m)| - |\dom(m)| \]

We calculate the variant at the point where we consider terminating the loop, immediately after the concrete matching step.  If the previous iteration involved killing one or more $!$-boxes, $V_L\setminus\dom(m)$ will decrease (by the combined size of the $!$-boxes).  If the previous iteration involved expanding a $!$-box, $|\dom(m)|$ will increase (by the size of the $!$-box, excluding any bare wires).  $V_L\setminus\dom(m)$ is clearly bounded below by $0$, and $|\dom(m)|$ is bounded below by $|V_G|$, by global invariant \ref{enum:g-inv-inj}, so $n$ is bounded below by $0$ and decreases with every iteration.  Hence the algorithm terminates.


\subsection{Concrete Matching Preconditions}

We need to show that the preconditions of the concrete matching step are satisfied whenever it is reached.

\begin{enumerate}[label=(\roman*),ref=(\roman*)]
  \item $L$ and $G$ are both normalised

  We normalise the graphs at the start.  We never alter $G$ after that.  Because $!$-boxes are open subgraphs, we always remove or copy an entire wire at once, which ensures that $L$ remains normalised throughout.
  \item The match state satisfies the global invariants

  Only the concrete matching step alters $m$ or $P$ in any way, and this maintains the global invariants.
  \item $U_C\cap\dom(m) = U_W\cap\dom(m) = U_N\cap\dom(m) = \varnothing$

  We only ever add unmatched vertices to these sets.
  \item Each vertex in $U_W$ is adjacent to something in $U_N\cup P_S$

  This follows from openness of $!$-boxes, which (together with normalisation) ensures that any concrete wire-vertex that is not on a circle or bare wire must be adjacent to a concrete node-vertex.  The first time concrete matching happens, all such node-vertices are put in $U_N$.
  
  Since all non-bare-wire, non-circle concrete vertices are added to $U_W$ each time, and so added to $\dom(m)$ by the concrete matching step, on subsequent iterations of concrete matching any wire-vertex $v$ in $U_W$ must have arisen from the expansion of a $!$-box (since this is the only way concrete vertices can be added).  If $v$ is adjacent to a node-vertex that came from the $!$-box expansion, that node-vertex will be added to $U_N$.  Otherwise, it must be adjacent to a node-vertex in $\dom(m)$, and hence in $P$ (by the postconditions of the concrete matching step).  Then this node-vertex (being adjacent to the expanded $!$-box) will be added to $P_S$.
  \item If $v \in P$ and $N_G(m(v)) \subseteq \im(m)$, $v \in P_S$ (everything that is completely matched and in $P$ is also in $P_S$)

  The first time we run the concrete matching step, $P$ is empty.  After each concrete matching step, the postconditions ensure that there is no completely matched vertex in $P$.  As the rest of the algorithm does not add anything to $P$ or alter $m$, this is maintained throughout.
  \item If $v \in U_W$ and $w$ is a wire-vertex adjacent to $v$ in $L$, $w \in U_W$.

  This is guaranteed by openness of $!$-boxes: we only ever add entire wires to $U_W$.
  \item If $v \in \dom(m)$ is a wire-vertex, then any vertices it is adjacent to are also in $\dom(m)$

  This is guaranteed by the postcondition \ref{enum:il-postcond-all-matched} in combination with precondition \ref{enum:il-inv-wire-uw} and the fact that $m$ is only altered by the concrete matching step.
  \item $U_C$ contains only circles, and $U_W$ contains no wire-vertices that are in circles

  This is implicit in the step where we populate $U_W$ and $U_C$.
\end{enumerate}

\subsection{Correctness}

At the end of the algorithm, everything that was ever in the concrete part of the graph except for bare wires has been added to $m$ by the concrete matching step (postcondition \ref{enum:il-postcond-all-matched}, together with the fact that we always add everything concrete and unmatched to $U_W$, $U_C$ and $U_N$, except bare wires).  We have no more $!$-boxes, so the entire graph is concrete.  $P$ is also empty.  We matched bare wires at the last stage, so $m$ is in fact total on $L$.  So, by the argument in section \ref{sec:match-state}, $m$ is a valid match from the final state of $L$ to $G$.  Since we have only ever performed valid $!$-box operations on $L$, this is a valid instance of $L$, and hence we have found a valid matching from $L$ to $G$.

\subsection{Completeness}

\TODOinline{all possible matchings are found}

\section{Improved Pattern Wrapper: Component-wise !-box Expansion}

The previous algorithm greedily matches all concrete vertices in each iteration. However, in a situation where different concrete components of a graph are connected only by a path containing !-boxes, this could produce many spurious match candidates that only get ruled out when the connecting !-box is killed or expanded. The following algorithm removed this problem by doing concrete matching component-wise, only moving to the next component once all !-boxes connected to the currently matched concrete components have been expanded.

\begin{center}
  \includegraphics[height=16cm]{component_pattern_part.pdf}
\end{center}

\section{Alternate Wrapper: Partial (Non-strict) Graph Matching}

The advantage of the concrete/wrapper separation is it allows easy implementation of alternative pattern graph paradigms. The one described in this section is strictly weaker than $!$-box matching, yet in many cases could perform better, as the pattern expansion does not require an indeterminism. That is, it is uniquely determined by a single run of the concrete matching algorithm.

A partial string graph $(L, \PN(L))$ is a string graph with a marked subset of node-vertices $\PN(L) \subseteq \N(L)$ that are allowed to be matched partially. A partial string graph rewrite rule is a normal string graph rewrite rule $(L,R)$ with a surjection $p: \PN(L) \twoheadrightarrow \PN(R)$. \textit{Reversible} PSG rules are those where $p$ is a bijection.

A PSG match is a normal match, with the exception that local iso need only be satisfied for node-vertices in $\N(L) - \PN(L)$. Instantiation of a PSG is done by adding new edges with boundaries for every unmatched edge in the neighbourhood of $m(n)$ for all $n \in \PN(L)$. For rule instantiation, the new edge/boundary pairs are added both the the neighbourhood of $n \in \PN(L)$ and $p(n) \in \PN(R)$.

Matching can be done as a wrapper for the concrete matching algorithm.

\begin{center}
  \includegraphics[height=12cm]{nonstrict.pdf}
\end{center}

The result is the same as DPO rewriting for more general spans $L \leftarrow I \rightarrow R$ where $I$ contains the shared boundary of $L$ and $R$ along with some node-vertices. For non-reversible rules, the map $I \rightarrow R$ does not need to be mono.

\end{document}
